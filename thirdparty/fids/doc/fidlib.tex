% -*-latex-*-
% Document name: /u/macleod/dfile/fidlib/fidlib.tex
% Creator: Rob MacLeod [macleod@vissgi.cvrti.utah.edu]
% Creation Date: Mon Oct 28 13:06:42 1996
%%%%%%%%%%%%%%%%%%%%%%%%%%%%%%%%%%%%%%%%%%%%%%%%%%%%%%%%%%%%%%%%%%%%%%
\documentclass[11pt]{article}
\textheight 8.7in
\textwidth 6.5in
\topmargin -.5in
\oddsidemargin 0in
\setlength{\parskip}{\smallskipamount}
\newcommand{\eg}{{\em eg.,}}
\newcommand{\ie}{{\em i.e.,}}
\newcommand{\etc}{{\em etc.,}}
\newcommand{\etal}{{\em et al.}}
\newcommand{\degrees}{{$^{\circ}$}}
\newcommand{\splitline}{\begin{center}\rule{\columnwidth}{.7mm}\end{center}}
%%%%%%%%%%%%%%%%%%%%%%%%%%%%%%%%%%%%%%%%%%%%%%%%%%%%%%%%%%%%%%%%%%%%%%
\newcommand{\X}[1]{#1\index{#1}}
\begin{document}

\title{Fid File Library}
\author{Rob MacLeod}

\maketitle

\section{Introduction}

\subsection{What is a Fids File?}

A fids file contains fiducial frame numbers from each lead of mapping
data.  The potentials are usually stored in .data files and there we have
fiducials that apply to the whole beat, that is, the same value applies to
every lead in the mapping array.  Eventually, the information in fids files
will migrate to the .data files too, but in the meantime---and thereafter
if experience is any guide---we have fids files.

A fids file is in ASCII format, and hence can be easily created by anything
from a text editor to awk, C, and Fortran programs.  The disadvantage is
that ASCII files are large and there is less control of format.  What this
means in practice is that the routines in the fids library may fail because
the file is not quite in the correct format.  The errors causing the
failure may be hard to find, harder than if this were a binary file.  But
this is the price of simplicity and easy access.

\subsection{ The Standard}

The standard format for the fids files is as follows:

\begin{verbatim}
 %% text header goes here %%
 num-time-series
 time-series-num  pakfilenum
 nleads fid1 fid2 fid3 .....
 1 10 20 30  ...
 2 10 20 31 ...
 ...
 time-series-num2 pakfilenum2 
 nleads fid1 fid2 fid3 .....
 1 10 20 30  ...
 2 10 20 31 ...
 ...
\end{verbatim}
%
and so on for multiple time series.  

The difference between {\tt time-series-num} and {\tt pakfilenum} is that 
{\tt time-series-num} is the running number within the file itself; 
it starts at one and increases for each time series in the file.  The 
values for pakfilenum is only really meaningful to link fid information to 
.pak files on the Vax and this is how I anticipate it being used.

The numeric values are in frame numbers relative to the full pack file,
starting at 1.  The ``fid1'', ``fid2'', \etc{}. are strings that tell us what
type of fiducials are stored in the file and it what order.  The legal
strings for this must include the following:

\begin{tabular}{|l|l|} \hline
\multicolumn{1}{|c|}{Text} &
\multicolumn{1}{|c|}{Contents} \\ \hline
pon	 & P-onset \\
poff	 & P-offset \\
qon	 & Q-onset \\
rpeak	 & Peak of the R wave \\
soff	 & S-offset == end of QRS \\
tpeak	 & Peak of the T wave \\
toff	 & End of the T wave \\
act	 & Activation time  \\
actplus	 & Activation time Max + slope \\
actminus & Activation time Max - slope \\
rec	 & Recovery time \\
recplus	 & Recovery time Max + slope \\
recminus & Recovery time Max - slope \\
\hline
\end{tabular}

The standard file extension for these files is ``.fids''.

Comments are designated by the symbols \#, \%, or ! and can appear anywhere
in the file.

All values are assumed to be in frame numbers {\bf beginning with 1!!!}.

\subsection{What is this library?}

This library offers some routines to manage fids files, especially reading
existing fids files into a data structure defined for this purpose.  I hope
they make programs a little easier to write and perhaps exert some gentle
pressure towards a standard approach to these files.  At least it makes my
life easier as I have written several program using the library and will
probably have more in the near future.

\section{The Library}

The code for the library will normally be kept in /usr/llocal/dfile/fids,
with a local copy maintained and updated by Rob (macleod@cvrti.utah.edu).

The library itself is called /usr/llocal/lib/libfids.a


\subsection{The data structure}

The contents of the fids.h file reveal the latest format of the data
structure. At present this looks as follows: 

\begin{verbatim}
enum fidtype {FID_PON, FID_POFF, FID_QON, FID_RPEAK, FID_SOFF,
	      FID_TPEAK, FID_TOFF, FID_ACTPLUS, FID_ACTMINUS, FID_ACT, 
	      FID_RECPLUS, FID_RECMINUS, FID_REC };

typedef struct Fiducials
{
    long ponframe;
    long pofframe;
    long qframe;
    long rpeak;
    long sframe;
    long tpeak;
    long tframe;
    long actpframe; /*** Activation frame max + dv/dt***/
    long actmframe; /*** Activation frame  max - dv/dt ***/
    long actframe; /*** Actual or best activation frame ***/
    long recpframe; /*** Recovery frame max + dv/dt ***/
    long recmframe; /*** Recovery frame max - dv/dt ***/
    long recframe; /*** Actual or best recovery frame ***/
    long aridur; /*** Difference between activation and recovery ***/
} Fiducials;

typedef struct FidList /*** List of set of fiducials. ***/
{
    long tsnum;	/*** Fiducial series number from the fid file starting at 1.*/
    long pakfilenum; /*** Pak file number associated with fids. ***/
    long numfids;
    Fiducials *fids;
} FidList;
\end{verbatim}

\subsection{The Functions}

\subsubsection{File management}

I list these first as they are a key component of the functions described
below. \\
%
\begin{tabular}{|l|p{4in}|} \hline
\multicolumn{1}{|c|}{Function Name} &
\multicolumn{1}{|c|}{Purpose} \\ \hline
ReadComments & To read over any comments or blank line until we hit the
next line of useful information in the file; this makes the inclusion of
copious comments very easy to deal with when reading the file.\\
QComment & Returns a logical value which indicates whether or not a line is
considered a comment (\#, \%, and ! are the characters scanned for) \\
FindFidSeries & Move to a desired time series in the fid file; this can be
specified in terms of running time series numbers, as well as pakfile
numbers.\\ 
SkipFidSeries & Used most internally, this routine jumps over the current
time series and places the file read pointer at the line in which the number
of leads and fid labels reside. \\
SetFid & Also used internally to link fid type values, which are numeric,
with the elements of the data structure in which the information eventually
belongs. \\
GetNumFidSeries & Returns the number of time series in a particular file.\\
FindNumFids & Returns the number of valid fiducial types that are in a
passed FidList data structure; valid in this context means that fid values
are larger than 0.\\ \hline
\end{tabular}

\subsubsection{Reading a fid file}

Fid files are organized by time series and in each time series, we have a
matrix organized by lead number (the rows) and fiducial type (the
columns).  The number of rows has to equal the number of leads in the
associated data file but is otherwise unbounded;  the number of fiducials
is completely arbitrary.  The data structure that I defined to contain this
information is organized similarly, with a single instance of FidList for
each time series and a single instance of Fiducials for each lead.  So the
typical arrangement is to have an array of type FidList, and within each
element, an array of type Fiducials.  Within each element of the Fiducials
array is an entry for every allowed type of fiducial.

Reading a fids file reflects this structure.  We read in a single FidList
worth of data at a time and either process the results right away or store
the results in an array of type FidList for later processing.  This is the
purpose of the {\tt ReadFidList} function, which takes the times series and
pakfile numbers to find the right time series and returns a single FidList
structure.  

Here is an example in which the contents of a number sequence (numseq code
compliments of Ted Dustman) is used to drive the search for specific time
series of fid values.  The Boolean qbypaknum determines whether pakfile
numbers of time series numbers drive the search and assigns each element of
the number list to the appropriate variable.  {\tt ReadFidList} looks at
both {\tt tsnum} and {\tt pakfilenum} and chooses based on which is
non-zero and passes the directions to {\tt FindFidSeries} to get to the
correct point in the fids file.
%
\begin{verbatim}
    while (NumSeqNext(tsnumseq, &intval))
    {
	if ( qbypaknum )
	    pakfilenum = (long) intval;
	else
	    tsnum = (long) intval;
	printf(" Going to read fids with tsnum = %d and pakfilenum = %d\n",
	       tsnum, pakfilenum);
	error = ReadFidList( report_level, fidsfilename,
			     tsnum, pakfilenum, 
			     &fidlist[numfidseries++] );
	if ( error < 0 )
	{
	    printf(" Error return from reading fids\n");
	    exit(ERR_FILE);
	}
    }
\end{verbatim}

Within the {\tt ReadFidList} function is a call to the function {\tt
ReadFidSeries}, which then reads the fiducials values themselves into the
FidList.Fids structure.

\subsubsection{Writing a fids file}

Fortunately, writing a fid file is typically a lot easier than reading
one.  The fidlib does provide one routine for writing fid files, called
{\tt WriteFidList} and its arguments mimic those of {\tt ReadFidList}.

\section{Sample/Utility Programs}

\paragraph{smap: }  {\em smap\/} is a graphics program that displays a
.data file as a set of scalar leads but will also display fiducial values
it gets from a fids file, and allow you to edit and save these values in a
new fids file.

\paragraph{fidtodata:  } This program converts fids files to .data files
and thus allows you to view them with {\em map3d}.

\paragraph{makedata: } makedata creates .data files from .pak files on the
Vax and now also permits fids file data to be appended to the .data file.
Run this program whenever you need to get at pak files with {\em map3d\/}
and also wish to have .fids file data visible.

\paragraph{dartofids.awk: } An awk script that can read Marc Fuller's dar
files and convert them to fids files.

\paragraph{fidtopot.awk: } An awk script that takes fids files and makes
.pot files from them.

\end{document}

