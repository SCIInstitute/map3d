\documentclass[letterpaper,11pt]{article}
\usepackage{alltt}
\newcommand{\subfection}[2]{\subsection*{\ttfamily#2}\addcontentsline{toc}{subsection}{\ttfamily #1}}
\newcommand{\btt}{\begin{alltt}}        % Begin `typed' text.
\newcommand{\ett}{\end{alltt}}                  % End `typed' text.
\newcommand{\itt}[1]{{\ttfamily #1}}
\textheight 8.5in
\textwidth 6.5in
\topmargin -.5in
\oddsidemargin 0in
\setlength{\parskip}{\smallskipamount}

\begin{document}

\title{The Fiducial Info Library}
\author{Ted Dustman}
\maketitle
\pagenumbering{roman}
\tableofcontents 
\pagebreak
\pagenumbering{arabic}

\section{Introduction}

Libfi is a library of routines for obtaining information about fiducial
types from a fiducial database.

Each fiduical type has the following information associated with it: a
name, a type, a short description, and several pieces of color information.
The routines in this library fetch this information for you.

There are 2 sets of routines.  One set for C programmers and one set for
Fortran programmers.  There is only one library however, which holds
routines for both languages.

Currently, this library exists only on the Silicon Graphics workstations.

\section{Fiducial Type Information}

C programmers communicate with the library routines via the 
\itt{FiducialInfo} structure.  This structure contains all information for
one fiducial type.  The \itt{FiducialInfo} structure is used to pass information to
the routines and to recieve information from the routines.  Here is the
declaration of the \itt{FiducialInfo} structure:

\btt
typedef struct FiducialInfo \{
    const char *name;       /* Fiducial name. */
    short type;             /* Fiducial type. */
    const char *label;      /* Short descrip of the fiducial. */
    const char *xColorName; /* The X11 color name for the fiducial. For
                               use with X11. */
    short red, green, blue; /* The RGB color values for the
                               fiducial. For use with X11 or gl's rgb
                               mode. */
    XColor xColor;          /* X11 color value. For use with X11. */
    Colorindex glColorIndex;/* The gl color index for the
                               fiducial. For use with gl's colormap
                               mode. */
\} FiducialInfo;
\ett

This declaration is in the header file /usr/local/include/fi.h.

Fortran programs don't use the \itt{FiducialInfo} structure.  They obtain the
same information using the Fortran specific calls.  Note however that the
Fortran routines do not support the X11 color model.

\section{Usage}

In general fiducial names are used to communicate with the users of your
programs via a command line or by way of a menu.  Symbols defining
fiducial types and their corresponding names can be found in the files 
\itt{fi.h}
for C programmers and \itt{fi\_f.h} for Fortran programmers.

Fiducial types are for your programs' internal use and for writing to and
reading from files.  Given a fiducial name you can obtain a fiducial type
via the C routine \emph{FI\_GetInfoByName} or the Fortran routine
\emph{fi\_fnametotype}. Users of your programs should never see the value of a
type nor should you ask your users to input a type value.  You should
communicate with your users via fiducial names.  In fact, as a programmer
you should have no need to know the actual value of a fiducial type.  Of
course, never hard code a fiducial value in your programs.

Fiducial labels can be used to communicate additional information about
fiducials to users of your programs.

The color information is used when your programs draw fiducials.  C support
is provided for the X11, gl color map model, and gl rgb model.  Fortran
support is provided for the gl color map and gl rgb models only.  Color
information for the X11 color model always comes from the default colormap.

\section{Libraries}

There are actually 4 basic versions of the fi library.  Each library is
distinguished by the type of color information it supports, if any.  There
is additionally a shared and non-shared version of each basic library.
Thus there are a total of 8 libraries.  The following table lists all
libraries: 

\begin{itemize}
        \item  libfi.a - A basic non-shared version of the library.  No color support at
all.

        \item  libfi.so - Shared version of above.

        \item  libfix11.a - Support for X11 color only.  Non-shared.

        \item  libfix11.so - Shared version of libfix11.a

        \item  libfigl.a - Support for gl color only.  Non-shared.

        \item  libfigl.so - Shared version of libfigl.a.

        \item  libfix11gl.a - Support for x11 and gl color.  Non-shared.

        \item  libfix11gl.so - Shared version of libfix11gl.a.
\end{itemize}

All versions of the library are in /usr/local/lib.

Well, I still haven't told the whole story about these libraries.  The 
above libraries are ``o32'' libraries.  There are also ``n32'' 
versions of the libraries.  These exist as shared libraries only.  
These can be found in /usr/local/lib32.  

\section{Compiling And Linking}

You must compile your program appropriately depending on which library you
will be using.

When using libfi.a or libfi.so you need not do anything special.  Just be
sure to \emph{\#include ``fi.h''} for a C program or fi\_f.h for a Fortran program.
Your compile command will look something like the following when using the
non-shared version of the library:

\begin{verbatim}
cc -o myprog myprog.c -I/usr/local/include /usr/local/lib/libfi.a
f77 -o myprog myprog.f -I/usr/local/include /usr/local/lib/libfi.a
\end{verbatim}

Your compile command will look something like the following when using the
shared version of the library:

\begin{verbatim}
cc -o myprog myprog.c -I/usr/local/include -L/usr/local/lib -lfi
f77 -o myprog myprog.f -I/usr/local/include -L/usr/local/lib -lfi
\end{verbatim}

When using other versions of the library your compile command must include
one or both of these options: -DFI\_USE\_X11 and/or -DFI\_USE\_GL.  These
options indicate the desired color support.

If you are using libfix11.a, libfigl.a, or libfix11gl.a you must also
include these options: -lX11 and/or -lgl.  Don't include these options if
using the shared versions of these libraries.

\section{C Routines}

This section provides a description of each C routine.  Unless otherwise
noted each routine is available in all versions of the library.

\subfection{FI\_Init}{int FI\_Init(int reportLevel);}
\itt{FI\_Init} initializes the fi library.  You must call this routine before
calling any others. The variable reportLevel determines if error messages will be
printed on subsequent calls to the library routines.  If reportLevel is 0
then no messages will be printed if errors occur.  If it is 1 then error
messages will be printed.

\subfection{FI\_Done}{int FI\_Done(void);}
\itt{FI\_Done} deallocates resources used by the FI library.  Call this routine
when you are through with the library.


\subfection{FI\_InitX11Colors}{int FI\_InitX11Colors(Display *display);}
This routine is available in libfix11.a and libfix11.so only.

\itt{FI\_InitX11Colors} allocates and initializes required X11 color resources.
Call \itt{FI\_InitX11Colors} if you are going to be drawing fiducials in the X11
window system.  Returns TRUE if the X11 color resources were initialized
correctly, FALSE otherwise.

You must call \itt{FI\_Init} before calling this routine.  If you call this
routine then you may not call \itt{FI\_InitGLColorMapMode} or
\itt{FI\_InitGLRGBMode}.  In other words you can work in only one color mode
at a time.

\subfection{FI\_InitGLColorMapMode}{int FI\_InitGLColorMapMode(void);}
This routine is available in libfigl.a and libfigl.so only.

\itt{FI\_InitGLColorMapMode} allocates and initializes needed gl color resources.
Call \itt{FI\_InitGLColorMapMode} if you are going to be drawing fiducials using
gl's colormap mode.  Returns TRUE if the gl color resources were
initialized correctly, FALSE otherwise.

You must call \itt{FI\_Init} before calling this routine.  If you call this
routine then you may not call \itt{FI\_InitX11Colors} or 
\itt{FI\_InitGLRGBMode}.  In
other words you can work in only one color mode at a time.

\subfection{FI\_InitGLRGBMode}{int FI\_InitGLRGBMode(void);}
This routine is available in libfigl.a and libfigl.so only.

Call \itt{FI\_InitGLRGBMode} if you are going to be drawing fiducials using gl's
RGB color mode.  Returns TRUE if the gl RGB colors were initialized
correctly, FALSE otherwise.

You must call \itt{FI\_Init} before calling this routine.  If you call this
routine then you may not call \itt{FI\_InitX11Colors} or
\itt{FI\_InitGLColorMapMode}.  In other words you can work in only one color
mode at a time.

\subfection{FI\_GetNumTypes}{int FI\_GetNumTypes(void);}
\itt{FI\_GetNumTypes} returns the total number of fiducial types in the database.

\subfection{FI\_GetInfoByName}{int FI\_GetInfoByName(FiducialInfo *fi);}
\itt{FI\_GetInfoByName} uses the name field of the FiducialInfo structure to
retrieve the other FiducialInfo information.  It returns FALSE if name is
not in the fiducial database, TRUE otherwise.  Fiducial names can be found
in fi.h.

Example:

\btt
FiducialInfo fi;
int result;
char fidName[80];
GetFidNameFromUser(fidName);
fi.name = fidName;
result = FI_GetInfoByName(&fi);
if (result == TRUE)
    DoSomethingWithFidInfo(fi);
else
    NoSuchFiducialNameError(fidName);
\ett

\subfection{FI\_GetInfoByType}{int FI\_GetInfoByType(FiducialInfo *fi);}

\itt{FI\_GetInfoByType} uses the type field of the FiducialInfo structure to
retrieve all the other information. It returns FALSE if type is not in
the fiducial database, TRUE otherwise.

Example:

\btt
FiducialInfo fi;
int result;
fi.type = FI_ACT;
result = FI_GetInfoByType(&fi);
if (result == TRUE)
    DoSomethingWidthFidInfo(fi);
else
    NoSuchFiducialTypeError(type);
\ett

\subfection{FI\_NameToType}{short FI\_NameToType(const char *name);}

\itt{FI\_NameToType} returns a fiducial type number given the fiducial's name.
Behavior is undefined if the fiducial name is invalid.

\subfection{FI\_IsValidType}{int FI\_IsValidType(short type);}
\itt{FI\_IsValidType} returns TRUE if the given type is a valid type, FALSE
otherwise.  This routine is really only useful in assertion statements and
other similar error checking code.

\subfection{FI\_IsValidName}{int FI\_IsValidName(const char *name);}
\itt{FI\_IsValidName} returns TRUE if the given name is a valid name, FALSE
otherwise. 

\subfection{FI\_First}{int FI\_First(FiducialInfo *fi);} 
\itt{FI\_First} and
its companion \itt{FI\_Next} are used to iterate through all entries in the
database.  Entries are returned in alphabetical order according to
fiducial name.  \itt{FI\_First} gets the first entry in the database.  It returns TRUE if
successful, FALSE otherwise. 
These routines are not reentrant so don't nest them!

\subfection{FI\_Next}{int FI\_Next(FiducialInfo *fi);}
\itt{FI\_Next} fetches the next entry in the database.  After calling 
\itt{FI\_First} to get the first database entry, call \itt{FI\_Next}
repeatedly to get the remainder of the database entries.  It returns
TRUE if successful, FALSE when all fiducial types have been visited
(fi will not contain valid information in this case).  

\section{Fortran Routines}
This section provides a description of each Fortran routine.  Unless otherwise
noted each routine is available in all versions of the library.

\subfection{fi\_finit}{integer function fi\_finit(reportLevel) \\* integer reportLevel}
\itt{Fi\_fnit} initializes the fi library.  You must call this routine before
calling any others. ReportLevel determines if error messages will be
printed on subsequent calls to the library routines.  If reportLevel is 0
then no messages will be printed if errors occur.  If it is 1 then error
messages will be printed.

\subfection{fi\_fdone}{integer function fi\_fdone()}
\itt{Fi\_fdone} deallocates resources used by the FI library.  Call this routine
when you are through with the library.

\subfection{fi\_finitglcolormapmode}{integer function fi\_finitglcolormapmode()}
This routine is available in libfigl.a and libfigl.so only.

\itt{Fi\_finitglcolormapmode} allocates and initializes needed gl color resources.
Call \itt{fi\_finitglcolormapmode} if you are going to be drawing fiducials using
gl's colormap mode.  Returns .true. if the gl color resources were
initialized correctly, .false. otherwise.

You must call \itt{fi\_init} before calling this routine.  If you call this
routine then you may not call \itt{fi\_initglrgbmode}.  In other words you can
work in only one color mode at a time.

\subfection{fi\_finitglrgbmode}{integer function fi\_finitglrgbmode()}
This routine is available in libfigl.a and libfigl.so only.

\itt{Fi\_finitglrgbmode} allocates and initializes needed gl color resources.
Call \itt{fi\_finitglrgbmode} if you are going to be drawing fiducials using gl's
RGB color mode.  Returns .true. if the gl RGB colors were initialized
correctly, .false. otherwise.

You must call \itt{fi\_init} before calling this routine.  If you call this
routine then you may not call \itt{fi\_finitglcolormapmode}  In other words you
can work in only one color mode at a time.

\subfection{fi\_fgetnumtypes}{integer function fi\_fgetnumtypes()}
\itt{Fi\_fgetnumtypes} returns the total number of fiducial types in the database.

\subfection{fi\_fnametotype}{integer function fi\_fnametotype(name, type) \\*
character*(*) name \\*
integer*2 type}
\itt{Fi\_fnametotype} finds a fiducial's type given its name.  Returns .false. if
the name is not in the fiducial database, .true. otherwise.

\subfection{fi\_ftypetoname}{integer function fi\_ftypetoname(type, name) \\*
integer*2 type \\*
character*(*) name}
\itt{Fi\_ftypetoname} finds a fiducial's name given its type.  Returns .false. if
the type is not in the fiducial database, .true. otherwise.

\subfection{fi\_fisvalidtype}{integer function fi\_fisvalidtype(type) \\*
integer*2 type}\itt{Fi\_fisvalidtype} returns TRUE if the given type is a valid type, FALSE
otherwise. 

\subfection{fi\_fisvalidname}{integer function fi\_fisvalidname(name) \\*
character*(*) name}
\itt{Fi\_fisvalidname} returns TRUE if the given name is a valid name, FALSE
otherwise.

\subfection{fi\_fgetlabel}{integer function fi\_fgetlabel(type, label) \\*
integer*2 type \\*
character*(*) label)}
\itt{Fi\_fgetlabel} finds a fiducial's label given its type.  Returns .false. if the
type is not in the fiducial database, .true. otherwise.

\subfection{fi\_fgetxcolorname}{integer function fi\_fgetxcolorname(type, xColorName) \\*
integer*2 type \\*
character*(*) xColorName}
\itt{Fi\_fgettxcolorname}, finds a fiducial's x color name given its type.
Returns .false. if the type is not in the fiducial database, .true. otherwise.

\subfection{fi\_fgetrgbvalues}{integer function fi\_fgetrgbvalues(type, red, green, 
blue) \\*
integer*2 type, red, green, blue}
\itt{Fi\_fgetrgbvalues} find a fiducial's RGB color values given the fiducial's
type.  Returns .false. if the type is not in the fiducial database, .true.
otherwise.

\subfection{fi\_fgetglindex}{integer function fi\_fgetglindex(type, glColorIndex) \\*
integer*2 type, glColorIndex}
This routine is available in libfigl.a and libfigl.so only.

\itt{Fi\_fgetglindex} finds a fiducial's gl color index given the fiducial's type.
Returns .false. if the type is not in the fiducial database,
.true. otherwise.

\subfection{fi\_ffirst}{integer function fi\_ffirst(type) \\*
integer*2 type} \itt{Fi\_ffirst} and its companion \itt{Fi\_fnext}are
used to iterate through all entries in the database.  Entries are
returned in alphabetical order according to fiducial name.  These
routines are not reentrant so don't nest them!

\itt{Fi\_ffirst} finds the first entry in the database and gets its type.  Returns
.true. if successful, .false. otherwise. 

\subfection{fi\_fnext}{integer function fi\_fnext(type) \\* integer*2
type} \itt{Fi\_fnext} gets the next entry in the database and
determines its type.  After calling \itt{fi\_ffirst} to get the first
database entry call \itt{fi\_fnext} repeatedly to get the remainder of
the database entries.  Returns .true.  if successful, .false.  if all
entries have been exhausted.

\end{document}

