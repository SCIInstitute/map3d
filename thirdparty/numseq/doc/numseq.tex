\documentclass[letterpaper,twoside,10pt]{article}


\begin{document}

% Definition of title page:
\title{
        Numseq Library
}
\author{
        Ted Dustman     % insert author(s) here
}
\date{April 12, 1996}   % optional

\maketitle

\section{Introduction}
The number sequence library is a small collection of routines that generate
a sequence of integer numbers from a compact string description of the
sequence.  For example, the string description "1-5,10,15,20-25" describes
the following integer sequence: 1,2,3,4,5,10,15,20,21,22,23,24,25.  The
library provides services for parsing the string description and then
returning each number of the sequence in turn.

\section{Number Sequence String Description}
The string description example given above demonstrates most of the syntax
of the string description.  That is, a string description consists of
number ranges separated by commas.  A number range is two numbers separated
by a dash or just a single number.  "1" is the smallest number that may
appear in a range.

A number range may go from low to high or high to low, e.g. "1-5" or "5-1".
The number sequence library will return numbers in the order specified,
either 1,2,3,4,5 or 5,4,3,2,1.

Number ranges may overlap. E.g. "1-10,8-12".

It turns out that a period ('.') may be used to separate number ranges.  If
this is the case then the first number of that range will be returned to
your program as a negative number.  Thus the sequence "1-3,10-12.14-16,20"
represents the sequence 1,2,3,10,11,12,-14,15,16,20.

Spaces are not permitted in a string description.

\section{Using The Library/Description Of Library Routines}
In your programs a number sequence in represented by a `NumSeq' object.
You create a number sequence object using the routine `NumSeqCreate'.  This
routine takes no parameters and returns a NumSeq object.  It returns NULL
if a NumSeq object could not be allocated:
\begin{verbatim}
#include "numseq.h"
NumSeq ns;
ns = NumSeqCreate();
if (ns == NULL) {
    /* ERROR */
}
\end{verbatim}
To compile a string description of a number sequence use the routine
`NumSeqParse'.  It takes as input a NumSeq object and a string to parse.
It returns 0 if the string could be compiled, non-zero otherwise.

\begin{verbatim}
if (NumSeqParse(ns, "1-5,95-100") != 0) {
    /* ERROR */
}
\end{verbatim}
You can call NumSeqParse any number of times for a NumSeq object.  This
routine generates an internal (internal to the NumSeq object, that is)
representation of the number sequence from the string description.  When
you call NumSeqParse on a new string it simply replaces the old
representation with the new.

Once the string description has been compiled you can retrieve the sequence
of numbers using one of two routines.  The first routine is `NumSeqNext'.
NumSeqNext takes two parameters.  The first is a NumSeq object and the
second is a pointer to an integer variable to contain the next number in
the sequence.  This routine returns FALSE when no more numbers remain in
the sequence:
\begin{verbatim}
int num;
while (NumSeqNext(ns, &num)) {
    /* Do something with `num' */
}
\end{verbatim}
The second routine for retrieving the number sequence is `NumSeqArray'.
This routines builds an integer array containing the number sequence.  It
also returns the number of elements in the array:
\begin{verbatim}
int *nsa, n, i;
NumSeqArray(ns, &nsa, &n);
if (nsa == NULL) {
    /* ERROR */
}
for (i=0; i<n; i++) {
    /* Do something with nsa[i] */
}
free(nsa);
\end{verbatim}
Note that the routine allocates the array but the you, the caller, are
responsible for deleting it when you are done using it.

When you are through using a number sequence object you should delete it
using `NumSeqDestroy':
\begin{verbatim}
NumSeqDestroy(ns);
\end{verbatim}
You may find a couple of other routines useful.  `NumSeqN' returns the number
of numbers in the sequence:
\begin{verbatim}
int n;
n = NumSeqN(ns);
\end{verbatim}
`NumSeqReset' resets the number sequence so that the next call to
NumSeqNext will return the first number in the sequence:
\begin{verbatim}
NumSeqReset(ns);
\end{verbatim}
\section{An Example}
Here is an example demonstrating most routines in the library:
\begin{verbatim}
#include "numseq.h"
#include <stdlib.h>
#include <stdio.h>

int main(int argc, char *argv[]) 
{
    NumSeq ns;
    int num, n, *na, i;

    if (argc != 2) {
            printf("Usage: nstest number-sequence-string\n");
        return 1;
    }

    ns = NumSeqCreate();

    if (NumSeqParse(ns, argv[1])) {
            printf("? Illegal sequence string\n");
        return 1;
    }
    
    printf("Retrieve numbers via NumSeqNext()...\n");
    n = NumSeqN(ns);
    printf("There are %d numbers in the sequence\n", n);
    while (NumSeqNext(ns, &num)) {
        printf("n = %d\n", num);
    }

    printf("Retrieve numbers via NumSeqArray()...\n");
    NumSeqArray(ns, &na, &n);
    printf("n=%d\n", n);
    for (i=0; i<n; i++) {
        printf("na[%d]=%d\n", i, na[i]);
    }
    free(na);

    printf("Resetting...\n");
    NumSeqReset(ns);
    printf("Retrieve numbers via NumSeqNext()...\n");
    while (NumSeqNext(ns, &num)) {
        printf("n = %d\n", num);
    }

    NumSeqDestroy(ns); 
}
\end{verbatim}
Assuming this program is named nstest.c you can compile it with this
command:

cc -o nstest nstest.c -I/usr/local/include /usr/local/lib/libnumseq.so
\section{Location Of Header And Library Files}
The file "numseq.h" can be found in /usr/local/include.  The file
"libnumseq.so" can be found in /usr/local/lib.
\end{document}
