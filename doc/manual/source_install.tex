% -*-latex-*-
% Document name: source_install.tex
% Creator: Bryan Worthen [worthen@sci.utah.edu]
% Last update: Thu May 19 12:35:15 2005 by Rob Macleod
%    - for version 6.3
%%%%%%%%%%%%%%%%%%%%%%%%%%%%%%%%%%%%%%%%%%%%%%%%%%%%%%%%%%%%%%%%%%%%%%
\subsection{Installing from source}
\label{sec:source-install}

We have tried to make installing \map{} from source as simple as possible.  
There are four steps:


\begin{enumerate}
  \item Download the source
  \item Download and install map3d's external dependencies
  \item Setup the make configuration
  \item Compile
\end{enumerate}

\subsubsection{Download Source Code}
You can get the \map{} source code from the \map{} download page at 
\htmladdnormallink{\texttt{http://www.sci.utah.edu/software/map3d.html}}
{http://www.sci.utah.edu/software/map3d.html}.  Login and work your way
to the \map{} version \version{} download page, and download 
map3d-source.tar.gz.

\subsubsection{Install Dependencies}
You will need to install \map{}'s dependencies: gtk+, gtkglext, etc. 
To do this, please refer to Section~\ref{sec:windows-install},
Section~\ref{sec:sgi-install}, Section~\ref{sec:linux-install},
or Section~\ref{sec:mac-install}.  

If you are running on a different
system, you will probably need to download and install these on 
your own; please refer to \htmladdnormallink{GTK website}{http://www.gtk.org}
and \htmladdnormallink{GtkGLExt website}{http://gtkglext.sourceforge.net/}.

\subsubsection{Configuring \map{}}
\map{} in addition to GTK+, \map{} has other dependencies that have been
developed in conjunction with \map{}, which were designed to be used 
with and independently of \map{}.  The make system shipped with \map{}
was designed to allow users to compile these libraries independently
of \map{} if they so choose.  However, most users will not use
these libraries for any other purpose except for running \map{}.

You should not have to change much in order to get \map{} to compile.
MatlabIO (one of \map{}'s dependents) needs to know where ZLIB is
(required, and should be installed after completing the previous section),
and \map{} needs to know where gtk is.  The included files show samples of
how this is to be done, and following are specific details.

\paragraph{MS Visual Studio users}
To configure MatlabIO, open Visual Studio, load \file{map3dtop/map3d.sln},
right-click on the MatlabIO project, and select properties.  Under
Configuration properties, click C/C++, select General, and add the
directory where \file{zlib.h} is.

To configure \map{}, right-click on the \map{} project, and select
properties.  Under Configuration properties, click C/C++, select General,
and add the directory where all the gtk-based header files are (if you
installed the map3d-environment from the website, they should all be in the
same place).  I.e., \verb|\local\map3d-environment\include|.  Each directory
should be semi-colon delimited.

While still under map3d property pages, select Linker, and under General,
add the directory where the libraries are; i.e.,
\verb|\local\map3d-environment\lib|.


\paragraph{Mac, Linux, and SGI users}
To configure both MatlabIO and \map{}, it is only necessary to
modify \file{map3dtop/Makefile.incl}.

To configure MatlabIO, change
\var{ZLIB\_INC} to the directory that contains \file{zlib.h}  Also change
\var{ZLIB\_LIB} to the directory that contains the zlib library.

To configure \map{}, edit \file{map3dtop/map3d/Makefile} and change
\var{GTKTOP}, \var{GTKLIB}, \var{GTKGL\_INC}, and \var{GTKGL\_LIB} to
appropriate values.  If \file{gtk.h} is in
\file{/usr/local/include/gtk-2.0/gtk}, and \file{libgtk-x11-2.0.so} (or
\file{.dylib} for mac users) is in \file{/usr/local/lib}, then
%
\begin{verbatim}
GTK_INC=/usr/local/include
GTK_LIB=/usr/local/lib
\end{verbatim}
%
Similarly, \var{GTKGL\_INC} and \var{GTKGL\_LIB} need to be set.
to include the dir that contains \file{gtkgl.h} and
\file{libgtkgl-x11-1.0.so} (or \file{.dylib}) respectively.  (-I goes in
front of the include directory, and -L goes in front of the lib directory,
respectively.)
    
\subsubsection{Compiling \map{}}
Compiling \map{} now should be as simple as entering the map3dtop directory
and running
%
\begin{verbatim}
make
\end{verbatim}

Remember for \map{} to run, you should probably add \map{} to your path.
If, when you run \map{} see errors like ``Cannot load library gtk-2.0.dll''
or ``Cannot map so libgtk-2.0.so'', then you need to add the gtk libraries
to your runtime path.  To do this on windows, follow the directions on the
windows install page, otherwise, set the \var{LD\_LIBRARY\_PATH}.
To do this, do the following:
%
tcsh users:
\begin{verbatim}
   setenv LD_LIBRARY_PATH GTK_LIB:$LD_LIBRARY_PATH
\end{verbatim}
%
or bash users:
%
\begin{verbatim} 
    export LD_LIBRARY_PATH=GTK_LIB:$LD_LIBRARY_PATH
\end{verbatim}
%

Substitute the values that \var{GTK\_LIB} is set to in the
\file{Makefile.incl}.  You might want to put this line in your \texttt{.cshrc}
or \texttt{.profile} file to avoid having to run this multiple times.


\paragraph{Contact}
If there are problems, feel free to contact
\htmladdnormallink{map3d@sci.utah.edu}{mailto:map3d@sci.utah.edu}

%%% Local Variables: 
%%% mode: latex
%%% TeX-master: "manual"
%%% End: 
