% -*-latex-*-
% Document name: linux_install.tex
% Creator: Bryan Worthen [worthen@sci.utah.edu]
% Last update: Wed Oct 20 20:00:00 2004 by J.R. Blackham
%    - release 6.2
%%%%%%%%%%%%%%%%%%%%%%%%%%%%%%%%%%%%%%%%%%%%%%%%%%%%%%%%%%%%%%%%%%%%%%
\subsection{Linux Installation Instructions}
\label{sec:linux-install}

The Linux installation is relatively straightforward.  You'll need to 
download \map{}'s dependencies, then download \map{} itself.

\subsubsection{Linux Dependencies}

There are two phases to this part.  First we need to get GTK+ and its
dependencies.  The easiest way to do this is from your distribution's 
installation CDs, or you can download the RPMs at www.rpmfind.net.  

To get the dependencies from your distribution, run the Package Manager
(Add/Remove Applications, configure-packages or something of that sort).
Search for gtk, and install gtk2 (if you can't find that directly, then 
installing the gnome environment will take care of it).

To get the dependencies from the internet, navigate your favorite browser
to \htmladdnormallink{\texttt{http://www.rpmfind.net}} {http://www.rpmfind.net}, and
search for \texttt{gtk-2}.  Try to find one that matches your
distribution (redhat, mandrake, etc.). \map{} requires gtk2-2.4 or greater.

%If your GTK version is 2.0.6 or 2.2.1 (you can find out by
%looking at \texttt{gtkversion.h} which will be where you installed gtk
%(normally \texttt{/usr/include/gtk-2.0/gtk/gtkversion.h}), and look for
%\verb|GTK_MAJOR_VERSION|, \verb|GTK_MINOR_VERSION|, and
%\verb|GTK_MICRO_VERSION|.  There will be numbers on the same lines as each
%of these, and if you put them together it will be something like 2.2.1).
%If you are using one of these versions download
%\texttt{gtkglext-linux-2.2.1.tar.gz} or \texttt{gtkglext-linux-2.0.6.tar.gz} from
%the \map{} \htmladdnormallink{\texttt{download page}}
%{http://www.sci.utah.edu/software/map3d.html} and follow these
%instructions:
%
%\begin{verbatim}
%    cd <download directory>
%    gunzip gtkglext-linux-<version>.tar.gz
%    tar xf gtkglext-linux-<version>.tar
%    cp libgtkglext-x11-1.0.so.0 /usr/local/lib
%    cp libgdkglext-x11-1.0.so.0 /usr/local/lib
%\end{verbatim}

%You can copy them to some directory other than \texttt{/usr/local/lib} if
%you wish.

The next part is to download gtkglext, the library that supports OpenGL for
GTK widgets.  Some distrubutions come with gtkglext.
If yours doesn't, you will need to download the gtkglext source and
compile it yourself (don't worry---if your gtk is properly set up, this
will be very easy).  Download the sources from Source Forge
\htmladdnormallink{\texttt{http://sourceforge.net/projects/gtkglext}}
{http://sourceforge.net/projects/gtkglext/} and follow these instructions:
%
\begin{verbatim}
    cd <download directory>
    gunzip gtkglext-1.0.6.tar.gz
    tar xf gtkglext-1.0.6.tar
    cd gtkglext-1.0.6
    configure
    make
    make install
\end{verbatim}

If you don't want these to end up in \texttt{/usr/local/lib,} you need to
%
\begin{verbatim}
    configure --prefix=<dir>
\end{verbatim}
%
where \texttt{dir} is where to put the libraries. 

\subsubsection{Linux Executable}

Download the \texttt{map3d-6.5-linux.tar.gz} file from the \map{} download page
and unzip it to a directory of your choice.  We will call that
\texttt{RUN-DIR}. This is the directory from which you will run \map{}.

To run \map{}, you will need to make sure that all the libraries are in
your \verb|LD_LIBRARY_PATH| environment variable.  For this we will assume
that your gtk libraries are in \texttt{/usr/lib} and your gtkglext
libraries are in \texttt{/usr/local/lib}.  Do the following:
%
tcsh users:
\begin{verbatim}
   setenv LD_LIBRARY_PATH /usr/local/lib:$LD_LIBRARY_PATH
\end{verbatim}
or\\
%
bash users:
\begin{verbatim} 
    export LD_LIBRARY_PATH=/usr/local/lib:$LD_LIBRARY_PATH
\end{verbatim}

you might want to put this line in your \texttt{.cshrc} or
\texttt{.profile} file to avoid having to run this multiple times.

%%% Local Variables: 
%%% mode: latex
%%% TeX-master: "manual"
%%% End: 
