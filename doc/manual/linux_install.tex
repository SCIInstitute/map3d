% -*-latex-*-
% Document name: linux_install.tex
% Creator: Bryan Worthen [worthen@sci.utah.edu]
% Last update: Wed Oct 20 20:00:00 2004 by J.R. Blackham
%    - release 6.2
%%%%%%%%%%%%%%%%%%%%%%%%%%%%%%%%%%%%%%%%%%%%%%%%%%%%%%%%%%%%%%%%%%%%%%
\subsection{Linux Installation Instructions}
\label{sec:linux-install}

The Linux installation is relatively straightforward.  You'll need to 
download \map{}'s dependencies, then download \map{} itself.

\subsubsection{Linux Dependencies}

Basically, here you need to install Qt's development distribution.The easiest way to do this is from your distribution's 
installation CDs, or you can download the RPMs at www.rpmfind.net.  

To get the dependencies from your distribution, run the Package Manager
(Add/Remove Applications, configure-packages or something of that sort).
Search for qtk, and install gtk2 (if you can't find that directly, then 
installing the gnome environment will take care of it).

To get the dependencies from the internet, navigate your favorite browser
to \htmladdnormallink{\texttt{http://www.rpmfind.net}} {http://www.rpmfind.net}, and
search for \texttt{qt}.  Try to find one that matches your
distribution (redhat, fedora, etc.). \map{} requires qt-5.4 or greater.

\subsubsection{Linux Executable}

We do not presently support Linux binaries.  Building \map() from source
is possible.  Download the \map{} source tarball file from the \map{} download page
and unzip it to a directory of your choice.

These instructions are not complete, but the gist is: Go to the \map{} directory, run qmake, and make.
The \map{} executable will be in client/release/map3d.

%%% Local Variables: 
%%% mode: latex
%%% TeX-master: "manual"
%%% End: 
