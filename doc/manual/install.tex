% -*-latex-*-
% Document name: install.tex
% Creator: Rob MacLeod [macleod@vissgi.cvrti.utah.edu]
% Last update: Fri Sep 15 00:33:54 2000 by Rob MacLeod
%    - created
% Last update: Wed Sep 27 14:54:43 2000 by Rob MacLeod
%    - some fixes for bug reporting and spell checking
% Last update: Sun Sep 24 16:41:08 2000 by Rob MacLeod
%    - Version 5.0Beta release edition
% Last update: Mon Jul 23 13:28:30 2001 by Rob MacLeod
%    - release 5.2
% Last update: Fri Jun 21 13:32:48 2002 by Rob Macleod
%    - release 5.3
% Last update: Fri Jan 24 20:00:00 2004 by Bryan Worthen
%    - release 5.4
% Last update: Thu May 19 12:35:15 2005 by Rob Macleod
%    - for version 6.3
% Last update: Fri Aug 26 12:34:14 2005 by Bryan Worthen
%    - for version 6.4
% Last update: Fri Feb 16 08:32:27 2007 by Rob Macleod
%    - version 6.5
%%%%%%%%%%%%%%%%%%%%%%%%%%%%%%%%%%%%%%%%%%%%%%%%%%%%%%%%%%%%%%%%%%%%%%
\section{Installation}

\subsection{System requirements}

\map{} is written in standard C/C++ and uses the OpenGL and GTK+ libraries, 
both choices made to ensure broad portability of the program.

Note: As of \map{} 6.4 or greater, GTK+ version 2.4 or greater is required
to run \map{}.  Please update your dependencies according the the 
Windows, Linux, or Mac sections below.

  \begin{description}
    \item [All platforms: ]  OpenGL now comes standard on most systems.  
          Instructions on how to install GTK+ are described in detail below
          based on which platform you are installing.
          described below\mbox{}\\
          \begin{center}
          \begin{tabular}{|l|l|} \hline
            \multicolumn{2}{|c|}{Requirements for all systems.  } \\ \hline
          OpenGL libraries (GL and GLU) & version 1.1 +
     \footnote{A plus sign, ``+'', indicates ``or better''}\\
            OpenGL/window interface library (GLX)& \\
            \htmladdnormallink{GTK+}
            {http://www.gtk.org} 
            libraries and dependencies & version 2.4+ \\ \hline
          \end{tabular}
          \end{center}

            
    \item [Linux (i386): ] \map{} requires the OpenGL library, which is
              available 
          as the mesa library at \htmladdnormallink{www.mesa3d.org}
          {http://www.mesa3d.org} for any Linux platform.  For better
          performance, graphics cards from companies such as nVidia 
          \htmladdnormallink{(www.nvidia.com)}
          {http://www.nvidia.com/} usually provide OpenGL libraries.  \\
          \begin{center}
          \begin{tabular}{|l|l|} \hline
            \multicolumn{2}{|c|}{Requirements} \\ \hline
              Operating System & kernel 2.2.x\\
              Architecture & i386 (+ maybe PPC)\\
              Applications Binary Interface  &  libc2.1 \\
              \hline
            \multicolumn{2}{|c|}{Recommendations} \\ \hline
            Window system & XFree86 version 4.0 +\\
            Hardware & 3D graphics card (nVidia, 3dfx, ati)\\
            & 128 MB main memory \\ \hline
          \end{tabular}
          \end{center}
          
    \item [Windows: ] \mbox{}\\
          \begin{center}
          \begin{tabular}{|l|l|} \hline
            \multicolumn{2}{|c|}{Requirements} \\ \hline
              Operating System & W2K/NT4.0/9x\\
              Architecture & i386\\
              Applications Binary Interface  &  win32 \\
              \hline
            \multicolumn{2}{|c|}{Recommendations} \\ \hline
            Hardware & 3D graphics card (nVidia, 3dfx, ati)\\
            & 128 MB main memory \\ \hline
          \end{tabular}
          \end{center}

    \item [Mac OS X: ] \mbox{}\\
          \begin{center}
          \begin{tabular}{|l|l|} \hline
            \multicolumn{2}{|c|}{Requirements} \\ \hline
              Operating System & Mac OS 10.4(Tiger)\\
              Architecture & PPC or Intel\\
              \hline
            \multicolumn{2}{|c|}{Recommendations} \\ \hline
            Hardware & 3D graphics card (nVidia, 3dfx, ati)\\
            & Intel processor \\
            & $ge$512 MB main memory \\ \hline
          \end{tabular}
          \end{center}
          
   \item [SGI: ] we have dropped our support for SGI.  This platform, while
     so central to the origins of \map{} and OpenGL, is just not a
     significant desktop system in this day and age.  

  \end{description}
  
\subsection{General Installation}

Unfortunately, with our move to GTK+ for window support, it is not as easy
as past versions, which required just the download of an executable.  We
hope (in vain, perhaps) to be able to do that again in the future, but for
now we will attempt to make installation as easy as we can.  Simplified
instructions will be in a README file which comes with each package, and
are also listed below:

To download the software, use this URL
\htmladdnormallink{\texttt{www.sci.utah.edu/software/map3d.html}}
{http://www.sci.utah.edu/software/map3d.html}, and click on the
``Download'' button.  You'll need to sign into the SCI archive.  For
each of the installation instructions below, you can grab those file
from this page.


To test the installation, use the test files that accompany this
distribution.  Each has some script files included that show how to call
and execute \map{}.

% -*-latex-*-
% Document name: linux_install.tex
% Creator: Bryan Worthen [worthen@sci.utah.edu]
% Last update: Wed Oct 20 20:00:00 2004 by J.R. Blackham
%    - release 6.2
%%%%%%%%%%%%%%%%%%%%%%%%%%%%%%%%%%%%%%%%%%%%%%%%%%%%%%%%%%%%%%%%%%%%%%
\subsection{Linux Installation Instructions}
\label{sec:linux-install}

The Linux installation is relatively straightforward.  You'll need to 
download \map{}'s dependencies, then download \map{} itself.

\subsubsection{Linux Dependencies}

Basically, here you need to install Qt's development distribution.The easiest way to do this is from your distribution's 
installation CDs, or you can download the RPMs at www.rpmfind.net.  

To get the dependencies from your distribution, run the Package Manager
(Add/Remove Applications, configure-packages or something of that sort).
Search for qtk, and install gtk2 (if you can't find that directly, then 
installing the gnome environment will take care of it).

To get the dependencies from the internet, navigate your favorite browser
to \htmladdnormallink{\texttt{http://www.rpmfind.net}} {http://www.rpmfind.net}, and
search for \texttt{qt}.  Try to find one that matches your
distribution (redhat, fedora, etc.). \map{} requires qt-5.4 or greater.

\subsubsection{Linux Executable}

We do not presently support Linux binaries.  Building \map() from source
is possible.  Download the \map{} source tarball file from the \map{} download page
and unzip it to a directory of your choice.

These instructions are not complete, but the gist is: Go to the \map{} directory, run qmake, and make.
The \map{} executable will be in client/release/map3d.

%%% Local Variables: 
%%% mode: latex
%%% TeX-master: "manual"
%%% End: 

% -*-latex-*-
% Document name: windows_install.tex
% Creator: Bryan Worthen [worthen@sci.utah.edu]
% Last update: Fri Mar 26 20:00:00 2004 by Bryan Worthen
%    - release 6.0
%%%%%%%%%%%%%%%%%%%%%%%%%%%%%%%%%%%%%%%%%%%%%%%%%%%%%%%%%%%%%%%%%%%%%%
\subsection{Windows Installation Instructions}
\label{sec:windows-install}

The Windows installation is relatively straightforward.  You just need
to download and run the \map{} installation package.  \map{} can be run
from the Start menu or the Program Files directory.

%%% Local Variables: 
%%% mode: latex
%%% TeX-master: "manual"
%%% End: 

% -*-latex-*-
% Document name: mac_install.tex
% Creator: J.R. Blackham [blackham@sci.utah.edu]
% Last update: Fri Jun 30 15:00:00 2004 by J.R. Blackham
%    - release 6.5
%%%%%%%%%%%%%%%%%%%%%%%%%%%%%%%%%%%%%%%%%%%%%%%%%%%%%%%%%%%%%%%%%%%%%%
\subsection{Mac OS X Installation Instructions}
\label{sec:mac-install}

The Mac OS X installation is relatively straightforward.  You'll need to 
download \map{}'s DMG file, and then install it. Inside the DMG is a file
call \map{}. This is the executable and can be run from there.  All the
dependencies are included in the DMG.


%%% Local Variables: 
%%% mode: latex
%%% TeX-master: "manual"
%%% End: 

%% -*-latex-*-
% Document name: sgi_install.tex
% Creator: Bryan Worthen [worthen@sci.utah.edu]
% Last update: Fri Mar 26 20:00:00 2004 by Bryan Worthen
%    - release 6.0
%%%%%%%%%%%%%%%%%%%%%%%%%%%%%%%%%%%%%%%%%%%%%%%%%%%%%%%%%%%%%%%%%%%%%%
\subsection{SGI Installation Instructions}
\label{sec:sgi-install}

The SGI installation is relatively straightforward.  You'll need to 
download \map{}'s dependencies, then download \map{} itself.

\subsubsection{SGI Dependencies}
There are two phases to this part.  First we need to get GTK+ and its
dependencies.  The easiest way to do this is for SGI is to run a browser
with root access and go to SGI's freeware site at \htmladdnormallink
{\texttt{http://freeware.sgi.com}} {http://freeware.sgi.com}.  On that page
near the bottom there should be a link to a prereq calculator.  Click on
that link, and in the Freeware Product box, select \verb|fw_gtk2+| (make
sure you don't select \verb|fw_gtk+|).  Submit the query.  You should see a
list similar to this:
\begin{verbatim}
  fw_atk [relnotes] [prereqs] [install] 
  fw_expat [relnotes] [prereqs] [install] 
  fw_freetype2 [relnotes] [prereqs] [install] 
  fw_gettext [relnotes] [prereqs] [install] 
  fw_glib2 [relnotes] [prereqs] [install] 
  fw_libjpeg [relnotes] [prereqs] [install]
  fw_libpng [relnotes] [prereqs] [install]
  fw_libz [relnotes] [prereqs] [install]
  fw_pango [relnotes] [prereqs] [install]
  fw_tiff [relnotes] [prereqs] [install] 
\end{verbatim}
To install them, click on the install link one by one (and follow the 
instructions in the dialog boxes). 

IMPORTANT - when you install libz - it will mention something about a security
library being removed.  When you install libz, allow it to do this.  On the 
subsequent libraries, it will mention that the security package conflicts 
with libz, on these packages, have it continue without installing the
security package.

Install them in the order:

\begin{verbatim}
    gettext
    expat
    freetype2
    atk
    glib2
    pango
    libjpeg
    libtiff
    libz
    libpng
\end{verbatim}

After you've done all of these, click on the alphabetical link, and click on
the install buton that corresponds to \verb|libgtk2+-2.0.6|.

If for some reason, the prereq calculator isn't there or isn't working, go
to the alphabetical index and install the above in the order specified.


The next part is to download gtkglext, the library that supports OpenGL for
GTK widgets.  If your GTK version is 2.0.6 (you can find out by looking at
\texttt{gtkversion.h} which will be where you installed gtk (normally
\texttt{/usr/freeware/include/gtk-2.0/gtk/gtkversion.h}), and look for
\verb|GTK_MAJOR_VERSION|, \verb|GTK_MINOR_VERSION|, and
\verb|GTK_MICRO_VERSION|.  There will be numbers on the same lines as each
of these, and if you put them together it will be something like 2.0.6).
If you are using this version download \texttt{gtkglext-sgi.tar.gz} from
the \map{} download page
\htmladdnormallink{\texttt{http://www.sci.utah.edu/software/map3d.html}}
{http://www.sci.utah.edu/software/map3d.html} and follow these
instructions:

\begin{verbatim}
    cd <download directory>
    gunzip gtkglext-sgi.tar.gz
    tar xf gtkglext-sgi.tar
    cp libgtkglext-x11-1.0.so.0 /usr/local/lib
    cp libgdkglext-x11-1.0.so.0 /usr/local/lib
\end{verbatim}

You can copy them to some directory other than \texttt{/usr/local/lib} if
you wish.

If this doesn't work, you will need to download the gtkglext source and
compile it yourself (don't worry---if your gtk is properly set up, this
will be very easy).  Download the sources from Source Forge
\htmladdnormallink{\texttt{http://sourceforge.net/projects/gtkglext}}
  {http://sourceforge.net/projects/gtkglext} and follow these
  instructions:

\begin{verbatim}
    cd <download directory>
    gunzip gtkglext-1.0.6.tar.gz
    tar xf gtkglext-1.0.6.tar
    cd gtkglext-1.0.6
    configure
    make
    make install
\end{verbatim}


If you don't want these to end up in /usr/local/lib, you need to

\begin{verbatim}
configure --prefix=<dir>
\end{verbatim}

where \texttt{dir} is where you want to put the libraries (The libraries
will be in \texttt{dir/lib}).

\subsubsection{SGI Executable}

Download the \texttt{map3d-6.2-irix.tar.gz} file from the \map{} download
page and unzip it to a directory of your choice.  We will call that
\texttt{RUN-DIR}. This is the directory from which you will run \map{}.

To run \map{}, you will need to make sure that all the libraries are in
your \texttt{LD\_LIBRARY\_PATH} environment variable.  For this we will
assume that your gtk libraries are in \texttt{/usr/freeware/lib32} and your
gtkglext libraries are in \texttt{/usr/local/lib}.  Do the following:

\begin{verbatim}
    tcsh users:
    setenv LD_LIBRARY_PATH /usr/freeware/lib32:/usr/local/lib:$LD_LIBRARY_PATH
\end{verbatim}
%
or
%
bash users:
\begin{verbatim}
    export LD_LIBRARY_PATH=/usr/freeware/lib32:/usr/local/lib:$LD_LIBRARY_PATH
\end{verbatim}
%
you might want to put this line in your \texttt{.cshrc} or
\texttt{.profile} file to avoid having to run this multiple times.

%%% Local Variables: 
%%% mode: latex
%%% TeX-master: "manual"
%%% End: 

% -*-latex-*-
% Document name: source_install.tex
% Creator: Bryan Worthen [worthen@sci.utah.edu]
% Last update: Thu May 19 12:35:15 2005 by Rob Macleod
%    - for version 6.3
%%%%%%%%%%%%%%%%%%%%%%%%%%%%%%%%%%%%%%%%%%%%%%%%%%%%%%%%%%%%%%%%%%%%%%
\subsection{Installing from source}
\label{sec:source-install}

We have tried to make installing \map{} from source as simple as possible.  
There are four steps:


\begin{enumerate}
  \item Download the source
  \item Download and install map3d's external dependencies
  \item Setup the make configuration
  \item Compile
\end{enumerate}

\subsubsection{Download Source Code}
You can get the \map{} source code from the \map{} download page at 
\htmladdnormallink{\texttt{http://www.sci.utah.edu/software/map3d.html}}
{http://www.sci.utah.edu/software/map3d.html}.  Login and work your way
to the \map{} version \version{} download page, and download 
map3d-source.tar.gz.

\subsubsection{Install Dependencies}
You will need to install \map{}'s dependencies: gtk+, gtkglext, etc.  To do
this, please refer to Section~\ref{sec:windows-install},
Section~\ref{sec:linux-install}, or Section~\ref{sec:mac-install}.

If you are running on a different
system, you will probably need to download and install these on 
your own; please refer to \htmladdnormallink{GTK website}{http://www.gtk.org}
and \htmladdnormallink{GtkGLExt website}{http://gtkglext.sourceforge.net/}.

\subsubsection{Configuring \map{}}
\map{} in addition to GTK+, \map{} has other dependencies that have been
developed in conjunction with \map{}, which were designed to be used 
with and independently of \map{}.  The make system shipped with \map{}
was designed to allow users to compile these libraries independently
of \map{} if they so choose.  However, most users will not use
these libraries for any other purpose except for running \map{}.

You should not have to change much in order to get \map{} to compile.
MatlabIO (one of \map{}'s dependents) needs to know where ZLIB is
(required, and should be installed after completing the previous section),
and \map{} needs to know where gtk is.  The included files show samples of
how this is to be done, and following are specific details.

\paragraph{MS Visual Studio users}
To configure MatlabIO, open Visual Studio, load \file{map3dtop/map3d.sln},
right-click on the MatlabIO project, and select properties.  Under
Configuration properties, click C/C++, select General, and add the
directory where \file{zlib.h} is.

To configure \map{}, right-click on the \map{} project, and select
properties.  Under Configuration properties, click C/C++, select General,
and add the directory where all the gtk-based header files are (if you
installed the map3d-environment from the website, they should all be in the
same place).  I.e., \verb|\local\map3d-environment\include|.  Each directory
should be semi-colon delimited.

While still under map3d property pages, select Linker, and under General,
add the directory where the libraries are; i.e.,
\verb|\local\map3d-environment\lib|.


\paragraph{Mac, Linux, and SGI users}
To configure both MatlabIO and \map{}, it is only necessary to
modify \file{map3dtop/Makefile.incl}.

To configure MatlabIO, change
\var{ZLIB\_INC} to the directory that contains \file{zlib.h}  Also change
\var{ZLIB\_LIB} to the directory that contains the zlib library.

To configure \map{}, edit \file{map3dtop/map3d/Makefile} and change
\var{GTKTOP}, \var{GTKLIB}, \var{GTKGL\_INC}, and \var{GTKGL\_LIB} to
appropriate values.  If \file{gtk.h} is in
\file{/usr/local/include/gtk-2.0/gtk}, and \file{libgtk-x11-2.0.so} (or
\file{.dylib} for mac users) is in \file{/usr/local/lib}, then
%
\begin{verbatim}
GTK_INC=/usr/local/include
GTK_LIB=/usr/local/lib
\end{verbatim}
%
Similarly, \var{GTKGL\_INC} and \var{GTKGL\_LIB} need to be set.
to include the dir that contains \file{gtkgl.h} and
\file{libgtkgl-x11-1.0.so} (or \file{.dylib}) respectively.  (-I goes in
front of the include directory, and -L goes in front of the lib directory,
respectively.)
    
\subsubsection{Compiling \map{}}
Compiling \map{} now should be as simple as entering the map3dtop directory
and running
%
\begin{verbatim}
make
\end{verbatim}

Remember for \map{} to run, you should probably add \map{} to your path.
If, when you run \map{} see errors like ``Cannot load library gtk-2.0.dll''
or ``Cannot map so libgtk-2.0.so'', then you need to add the gtk libraries
to your runtime path.  To do this on windows, follow the directions on the
windows install page, otherwise, set the \var{LD\_LIBRARY\_PATH}.
To do this, do the following:
%
tcsh users:
\begin{verbatim}
   setenv LD_LIBRARY_PATH GTK_LIB:$LD_LIBRARY_PATH
\end{verbatim}
%
or bash users:
%
\begin{verbatim} 
    export LD_LIBRARY_PATH=GTK_LIB:$LD_LIBRARY_PATH
\end{verbatim}
%

Substitute the values that \var{GTK\_LIB} is set to in the
\file{Makefile.incl}.  You might want to put this line in your \texttt{.cshrc}
or \texttt{.profile} file to avoid having to run this multiple times.


\paragraph{Contact}
If there are problems, feel free to contact
\htmladdnormallink{map3d@sci.utah.edu}{mailto:map3d@sci.utah.edu}

%%% Local Variables: 
%%% mode: latex
%%% TeX-master: "manual"
%%% End: 


\subsection{Documentation}

This document should have reached you either as a pdf file or via the
\htmladdnormallink{\map{} web site}
{http://www.sci.utah.edu/software/map3d.html}.  If you would like more
copies or the latest version, go to the
same web site and look for the links under Documentation:\\

\htmladdnormallink{\texttt{www.sci.utah.edu/software/map3d.html}}
{http://www.sci.utah.edu/software/map3d.html}

\subsection{Bug reporting}

We want to hear your response to using \map{} and especially to learn about
any bugs you may find.  They may be features, rather than bugs, but if so,
we will be happy to hear your impressions.

To submit a bug report please send email to map3d@sci.utah.edu or point your
browser at
\htmladdnormallink{\texttt{software.sci.utah.edu/bugzilla}}
{http://software.sci.utah.edu/bugzilla/} (you will need to register your e-mail
address) with the following information:

\begin{enumerate}
  \item Type and version of the operating system and hardware you are using.
  \item Version of \map{}.
  \item Description of the bug/feature you encountered.
  \item Suggestions for fixing the bug or altering the program behavior.
\end{enumerate}


\subsection{How to reach us}

We have established an email address for \map{},
\htmladdnormallink{map3d@sci.utah.edu}{mailto:map3d@sci.utah.edu}, and
web pages within the website \htmladdnormallink{www.sci.utah.edu/ncrr}
{http://www.sci.utah.edu/ncrr} dedicated to \map{}.  There is also a
majordomo mailing list for \map{} users called
\texttt{map3d-users@sci.utah.edu}.  To subscribe to this list, send email to
\texttt{majordomo@sci.utah.edu} with the following in the message body\\
%
\begin{verbatim}
      subscribe map3d-users
\end{verbatim}

Please let us know how you use \map{} and how we can make it better for
your purposes.  We can only develop this program with continued
support and the best way to achieve this is to show that others use the
program and find it helpful.



%%% Local Variables: 
%%% mode: latex
%%% TeX-master: "manual"
%%% End: 
