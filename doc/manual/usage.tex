% -*-latex-*-
% Document name: usage.tex
% Creator: Rob MacLeod [macleod@vissgi.cvrti.utah.edu]
% Last update: September 2, 2000 by Rob MacLeod
%    - created
% Last update: Sun Sep 24 16:41:08 2000 by Rob MacLeod
%    - Version 5.0Beta release edition
% Last update: Mon Jul 23 13:28:30 2001 by Rob MacLeod
%    - release 5.2
% Last update: Fri Mar 1 20:00:00 2002 by Bryan Worthen
%    - release 5.3
% Last update: Fri Jun 21 13:33:12 2002 by Rob Macleod
%    - update still for 5.3
% Last update: Fri Jan 24 20:00:00 2003 by Bryan Worthen
%    - release 5.4
% Last update: Tue Mar 30 20:00:00 2003 by Bryan Worthen
%    - release 6.0
% Last update: Wed Jun 30 13:28:30 2004 by Bryan Worthen
%    - release 6.1
% Last update: Tue Oct 19 13:28:30 2004 by Bryan Worthen
%    - release 6.2
% Last update: Mon May 16 13:28:30 2005 by Bryan Worthen
%    - release 6.3
% Last update: Tue Jan 30 13:28:30 2007 by Bryan Worthen
%    - release 6.4
%%%%%%%%%%%%%%%%%%%%%%%%%%%%%%%%%%%%%%%%%%%%%%%%%%%%%%%%%%%%%%%%%%%%%%
\section{Usage} 
\label{sec:usage}

This version of \map{} provides two ways to load files.  The first is 
via the command line, which is described in this section.  The second is
via the files window (see Section~\ref{sec:fileswindow}).  You can also
launch \map{} with some command line options and then modify the associated
settings using interactive menus and the files window.  

This is a subset of \map{}'s usage:
\begin{verbatim} 
  map3d  -b -nw -nv
         -f geomfilename 
           -w
           -as xmin xmax ymin ymax 
           -al xmin xmax ymin ymax
           -at xmin xmax ymin ymax 
           -t time-signal-number 
           -c mesh colour
           -p scalar data (potentials) filename 
             -s num1 num2 
             -i increment 
             -ph maxpotval 
             -pl minpotval 
             -cs contour-spacing 
             -ps scaling_value 
             -ch channels-filename 
             -sl surfnum
             -ff fidfile
           -lm landmarks-filename
           -ll leadlinks-filename
           -lh 
\end{verbatim}

\subsection{Typical usage examples}
\label{sec:usage-examples} 

\noindent
Here are some typical examples of using \map{}:
%
\begin{itemize}
  \item Display the contents of a geometry file:
  \begin{verbatim}
      map3d
      map3d -f geomfilename.pts
      map3d -f geomfilename.fac
      map3d -f geomfilename.mat

  \end{verbatim}
  \vspace{-.2in}
  The first instance will run map3d and allow you to input files interactively
  (see Section~\ref{sec:fileswindow}).
  
  The first form with arguments reads only the node points (.pts file
  extension) while the second form also reads the connectivities from a
  .fac file and displays both mesh and nodes.  The third form assumes that
  a matlab geometry file (.mat extension) exists that contains both nodes
  and connectivities.  We describe all these forms of geometry files in
  Section~\ref{sec:geomfiles}.
  
  \item Map scalar values from a single time instant stored in a ``pot''
  file (described in Section~\ref{sec:potfiles}):
  \begin{verbatim}
      map3d -f geomfilename.fac -p datafilename.pot
  \end{verbatim}

  \item When there is a mapping required between the potentials and the 
  geometry, \eg{} when the order of values in the .pot and .pts file 
  are not identical, we require a channels file (see 
  Section~\ref{sec:leadfiles} for details of the channels files),
  \begin{verbatim}
      map3d -f geomfilename.fac -p datafilename.pot -ch thefile.channels
  \end{verbatim}
   
  \item To display a \emph{time series} of scalar values on the geometry, 
  the basic format is the same
  \begin{verbatim}
      map3d -f geomfilename.fac -p datafilename.mat
  \end{verbatim}
  \vspace{-.2in}
  with the time series stored in a datafile described in
  Section~\ref{sec:datafiles}.

  \item Geometry can also be stored in a matlab file (in a format 
  described in Section~\ref{sec:matlabgeom}).  The command 
  format is essentially unchanged
  \begin{verbatim}
      map3d -f geomfilename.mat -p datafilename.mat
  \end{verbatim}
  \vspace{-.2in}
  except that channel information is usually contained in the .geom 
  file and thus seldom needs to be specified explicitly.
  
  \item A time series data matlab file (.mat) contains a sequence of 
  potentials, as described in Section~\ref{sec:matlabdata}.
  To select a subset of the time series for display, append the 
  parameters \texttt{-s} and, optionally,  \texttt{-i}, for example,
  \begin{verbatim}
      map3d -f geomfilename.fac -p datafilename.mat -s 1 100 -i 2
  \end{verbatim}
  \vspace{-.2in}
  to select time instants 1 to 100 with an increment between instants 
  of 2 (\ie{} 1, 3, 5, 7, \ldots, 99).  

  \item Another way to describe a time series is through a series of .pot
  files that are numbered in sequence.  For example to read a sequence of
  the files \texttt{mapdata001.pot}, \texttt{mapdata002.pot},
  \texttt{mapdata003.pot}, \ldots{}  \texttt{mapdata009.pot}
  \begin{verbatim}
      map3d -f geomfilename.fac -p mapdata -s 1 9 
  \end{verbatim}
  
  \item Matlab geometry files can contain more than one geometry 
  so that we need to select a specific collection of nodes and 
  connectivities for the display, by means of an ``@'' suffix to the 
  geometry filename specification.  Calling 
  \begin{verbatim}
      map3d -f geomfilename.mat@2 -p datafilename.mat
  \end{verbatim}
  \vspace{-.2in}
  will select geometry \#2 from the file \texttt{geomfilename.mat}.

  \item Multiple instances of \texttt{-f} create multiple surfaces, which by
  default all appear in the same  window.  Adding the \texttt{-nw} option
  creates a separate window for each of the surfaces.  So a typical call
  would look like
  \begin{verbatim}
      map3d -f geomfile1.fac -p thedata1 -f geomfile2.fac -p thedata2 
  \end{verbatim}
  However, you can include all the regular features for each of the
  surfaces so that things can get much more complex.  For multi-surface
  displays, it is often better to use script files (see
  Section~\ref{sec:scripts}) below.

\end{itemize}

This version of \map{} provides an interactive means of 
specifying geometry numbers from a .geom file or time instants from a 
time series data file (see Section~\ref{sec:fileswindow}).

\subsection{Global Parameters}
\label{sec:usage-global} 

\noindent
The following general parameters affect the entire display:
\begin{description}
    
  \item[{\tt -b }] = open each individual window without borders
    placed within a master window that still has the usual borders.  To
    move or resize individual windows, hold the Alt (meta) key and use the left
    and middle mouse buttons, respectively.  Most of these can be anywhere on
    the command line.  Also, if you use {\tt -b} without any other arguments,
    \map{} will allow you to select the files interactively and add them to
    this master window.
    
    
  \item [{\tt -v}] = show current version of the program.  If this is
    the only argument, the program will exit.
    
%  \item[{\tt -vm}] = run program in video mode. This thickens lines,
%        and sets a reduced range of colours, for the purpose of making
%        videos from graphics output. 
%        This also initializes the video equipment in VISSGI\@. 
        
  \item[{\tt -nw}] = for multiple surfaces ({\em i.e.,} more than one
    set of points and triangles), place each surface data in a new
    window. By default, \map{} opens a single window for all
    surfaces.
    
  \item[{\tt -slw 0}] = do not show any legend windows at startup.

  \item[{\tt -fs interval}] = Sets the run-time interval between frames
    as accessed by the arrow keys.  Note that this is independent of the 
    {\tt -i} option below, which subsamples the data as it is being read in.
    This feature can be changed via the menus at runtime.
    
  \item[{\tt -fl}] = Allows keyboard navigation of frames to to loop,
    that is, to cycle from the end to the beginning of the dataset and 
    vice versa.
    This feature can be changed via the menus at runtime.
    
  \item[{\tt -nv}] = to NOT check validity of geometry files.  This can
    have a large impact on startup performance if \map{} needs to
    load large geometries.
    
  \item [{\tt -c colour}] = colour value to use on all
    surfaces for which there is no specific colour specification.  
    This option must be set before any surfaces are read, since the 
    same option sets the colors for individual surfaces.  See the 
    mesh-specific {\tt -c colour} below for colour examples. 
    
  \item [{\tt -bg colour}] = colour value to use as background of all
    windows for which there is no specific colour specification.  
    This option must be set before any surfaces are read, since the 
    same option sets the colors for individual windows.
    
  \item [{\tt -fg colour}] = colour value to use as text colour for all
    windows for which there is no specific colour specification.  
    This option must be set before any surfaces are read, since the 
    same option sets the colors for individual windows.
    
%  \item [{\tt -df filename}] = read the defaults file {\tt filename} for
%        start-up settings for the program.  This overrides any defaults
%        files read by the program (see below) or any defaults set within
%        the program. 

  \item [{\tt -if basefilename}] = base filename for any image files that
        are generated in this run of the program.

  \item[{\tt -dp datafile\_pathname}] = directs the search for data files
        accessed to another directory.  Using an alternate pathname, 
        you can override the
        original directory specification for the files and get them from,
        say, an optical disk.  This value can also be set with an
        environmental variable called {\tt MAP3D\_DATAPATH}, which you can
        set at any time before executing \map{}.  With this option, \map{}
        looks in {\tt datapath/filename}.

%        This value can also be specified in the
%        defaults file (see section~\ref{sec:defaults}) 

%  \item[{\tt -gp geomfile\_pathname}] = directs the search for geometry files
%        accessed via a .tsdf file to another directory.  The most common
%        case of this is when the name of the geometry file is stored in the
%        data file, usually without directory specification. This value can
%        also be specified in the
%        defaults file (see section~\ref{sec:defaults}) or via an
%        environmental variable called {\tt MAP3D\_GEOMPATH}, which you can
%        set at any time before executing \map{}.  You may specify this
%        option either globally for all surfaces by inserting is before the
%        first {\tt -f filename} in the command line, or separately for each
%        surface. 

      \item[{\tt -bgi image\_filename}] = An image file (in jpg or png) to
        display in the background.  Used in conjunction with -bgp.
      \item[{\tt -bgi x1 y1 z1 x2 y2 z2}] = Coordinates in world space to display
        the image specified with -bgp.  If unspecified, the image will fill the
        geometry window.
      \item[{\tt -ss}] = Sets the scale to be the same between all windows.
        Typically \map{} will position each window such that its geometries
        fill up the window, but this will enforce that all positioning is
        done the same.
      \item[{\tt -sr}] = Sets the amount of time (sample rate) between
        frames, for display purposes in the time series window.  This is on
        a time-unit basis; e.g., if the time unit is ns, and the sample
        rate is 3, the time series window will show that 3 ns passes with
        every frame advance.  Works in conjunction with the \emph{-tu}
        option.
      \item[{\tt -tu}] = Sets the time unit to display in the time series
        window.  Works in conjunction with the \emph{-sr} option.
      \item[{\tt -ss}] = Sets the scale to be the same between all windows.
        Typically \map{} will position each window such that its geometries
        fill up the window, but this will enforce that all positioning is
        done the same.
%  \item [{\tt -rl report\_level}] = set the report level for this execution
%        of \map{}.  The legal values are 0 (no reporting) to 4 (full
%        debug-level reporting).  
%       This values overrides the one set in the
%        default file, bit should be specified after the default file if the
%        {\tt -df} option is used.

%  \item [{\tt -vw xmin ymin}] = Set the lower left hand corner of the video
%        window to a particular spot on the screen.  Units are in screen
%        coordinates, same as for the {\tt -as} and {\tt -ts} commands
%        described below.

\end{description}
        
\subsection{Geometry specifications}
\label{sec:usage-geometry} 


The basis for display in \map{} is one or more geometry descriptions,
which are usually in the form of surfaces, but can also be a set of line
segments or tetrahedra; hence we can picture each set of nodes and
connectivities as a ``meta-surface'', which we generally refer to as a
``surface''.  For each such surface, \map{} needs the set of node locations
in three-dimensional space and usually some connectivity information that
defines the (meta) surface.  The geometries must exist in discrete form and
be stored in files that \map{} can read (see Section~\ref{sec:geomfiles} for
details of the file formats).  There is no provision at present for
analytically defined geometries.

To tell \map{} where to look for this geometry information, 
each occurrence of \texttt{-f} in the command line indicates that 
beginning of a new surface.  All parameters (except for global options)
that follow before the 
next occurrence of \texttt{-f} refer to the current surface.
%
\begin{description}
  \item[{\tt -f geometry-file} ]
                                %  or {\tt datafile.tsdf}]
    = filename of the geometry file(s) containing points and
    connectivity information.   Legal formats for the file specification 
    are: 
    \begin{enumerate}
      \item nodes (.pts) file will read and display only the 
        nodes from the geometry; no display of the potentials is 
        possible with just this information;
        
      \item triangles/tetrahedra (.fac/.tetra) file will read 
        \textbf{both} the connectivities and the nodes (provided
        both exist and share the same root filename);
        
      \item binary matlab geometry (.mat) file contains both nodes and 
        connectivity information and may also contain channel
        mapping.  At present, multi-surface geometry files must
	include a specific indication of the desired surface
	(\texttt{@surfnum}); otherwise, \map{} reads all surfaces
	in the file.
    \end{enumerate}
    Note: by specifying a root filename without any extension, \map{} will
    look for all valid geometry files and try and construct the most
    comprehensive set.  (It will do the same for data files as well.)
    Where there are multiple, potentially
    conflicting files with the same root, \eg{} \texttt{file.pts} and
    \texttt{file.geom}, \map{} will select binary over text files.  
    See Section \ref{sec:readgeom} for more details on the rules for
    specifying and reading geometry files.  
    
  \item[{\tt-w}] = place this and subsequent surfaces in a new window.
    This option will do nothing if the \texttt{-nw} option is set or if
    this is the first surface
    
  %    \item[{\tt -ac}] = do not shift the origin of the points to the centroid
  %    of the pointset but leave values as they are in the input geometry
  %    files. 
    
  \item[{\tt-fg colour}]
    = desired colour for the screen information of a particular 
    window, if this will be   
    specified as a red, green, and blue value triplet ranging from 
    0 to 255.  Some examples are:\\
                                % 
    \begin{center}
        \begin{tabular}{|l|l|} \hline
            255 0 0     & red \\
            0 255 0     & green\\
            0 0 255     & blue\\
            255 255 0   & yellow\\
            255 0 255   & magenta  \\
            0 255 255   & cyan \\
            255 255 255 & white  \\ \hline
        \end{tabular}
    \end{center}
    
  \item[{\tt-bg colour}]
    = desired colour for the background of a particular window,
    specified as a red, green, and blue value triplet.  
    See the {\tt-fg option} for examples.
    
  \item[{\tt-c colour}]
    = desired colour for the mesh of a particular surface,
    specified as a red, green, and blue value triplet.  
    See the {\tt-fg option} for examples.
    
    
  \item[{\tt-as xmin xmax ymin ymax}] = set the absolute location in
    pixels of the surface window most recently defined.  We
    assume an origin in the lower left corner of the screen and
    the typical full screen of an SGI workstation with a 19-inch
    monitor has 1280 by 1024 pixels.  This option is useful for
    setting consistent layout of windows, especially when
    there are multiple surfaces, each in its own window.
    
  \item[{\tt-al xmin xmax ymin ymax}] = set the absolute location in 
    pixels of the surface window most recently defined's colormap
    legend window.  There will be one of these windows per
    surface only if a valid data file is associated with it.
    
  \item[{\tt -slw 0}] = do not show the legend window for this surface.

  \item[{\tt -lh }] = Set the most recently defined surface's colormap
        legend window to have a horizontal instead of vertical layout.

  \item[{\tt -lm landmark\_filename }] = read from the file {\tt
      landmark\_filename} a set of coronary arteries, or any other
    landmark information stored as a series of points, with a radius
    associated with each.  See section~\ref{sec:lmfile} below for
    details.
    
  \item[{\tt -ll leadlinks-filename}] = file in {\em
      leadlinks\/} format containing a list of the node locations
    that correspond to a subset of the leads, \eg{} the lead
    locations on the torso surface that correspond to the standard
    ECG leads.  The point of identifying such leads is to display
    them with their own markings, either as spheres or with the
    lead number (typically not the same as the node number).  For
    more information, 
%        see the {\tt l} key-command in
%       Section~\ref{sec:control-keys}, which toggles the display of the
%       leads in the leadlink file and 
    see the menu options in
    Section~\ref{sec:control-menus} that determine the form of the display
    markings and Section \ref{sec:leadfiles} for more information on 
    leadlinks files.
    
%  \item[{\tt-cg colour}] = desired colour for the vectors in any grad
%        files read for this surface.  This is meaningful when the
%        amplitudes of the vectors have no intrinsic meaning (\eg{}
%        fiber-orientation vectors). \\ 
%        %      
%       \begin{tabular}{ll}
%          1 = red    & 4 = cyan\\
%          2 = green  & 5 = magenta  \\
%          3 = blue   & 6 = yellow \\
%          7 = white  &
%       \end{tabular}

%  \item[{\tt -ds }] = make this surface the dominant surface in a
%        master/slave relationship used in the program for moving surfaces
%        independently of one another.

%  \item[{\tt -dp datafile\_pathname}] = same as above but for this surface
%        only. 

%  \item[{\tt -gp geomfile\_pathname}] = same as above but for this surface
%        only. 


\end{description}

\subsection{Scalar Data parameters}
\label{sec:scalarparams} 

To display scalar data values on the geometry, we must specify the source 
of the data and how to link them to the geometry.  As with the 
geometry, all arguments specified between two occurrences of \texttt{-f} 
in the command line refer to the currently valid surface.  Within pairs of
\texttt{-f} options, there can be only a single instance of any of the
following options:
%
\begin{description}
    
  \item[{\tt-p potfilename}] = filename for the potential and
    current data files.  The legal file types for scalar data are:
    \begin{enumerate}
      \item Pot files (.pot) (see Section~\ref{sec:potfiles} are 
        text-based files which contain one file per time instance.
        
      \item matlab files (.mat) are also stored in binary format.  
        Matlab files can also contain multiple time series. 
        
    \end{enumerate}
    
    The \texttt{-s} option determines how many frames to load. 
    In the case of pot files, this controls which pot files to open 
    (If \texttt{-s} is omitted, it will only open the pot file specified). 
    For binary files, the \texttt{-s} option specifies the start
    and end frame numbers to be read from the file.  With no
    \texttt{-s} option, \map{} will read in all time instants from
    the file.  Note also that if you omit the extension, as with
    geometry files, it will try to match a .pot or .mat
    extension for you. \\

    For files with multiple time series (matlab or tsdf containers),
    you may specify the time series by the command line with the ``@'' 
    suffix appended to the filename followed by the time series index 
    within the file. \\

    {\em eg.,} \hspace{.2in}{\tt -p file.mat@1} reads the first time
    series and 
    {\em eg.,} \hspace{.2in}{\tt -p file.tsdfc@2} reads the second time
    series.
    
  \item[{\tt -s num1 num2}] = range of frame numbers to read.  If we are
    reading data from .pot or .grad files, \map{} appends each of the
    numbers between \texttt{num1} and \texttt{num2} 
    to the value of {\tt potfilename} to make complete pot filenames.
    However, you must run \map{} with the full pot filename (one of the 
    pot files in the series). \\ 
    {\em eg.,} \hspace{.2in}{\tt -p good-map001.pot -s 1 3}\hspace{.2in}
    expands to:\\
    \mbox{\hspace{1in}}
    {\tt good-map001.pot good-map002.pot good-map003.pot}\\
    If we are reading from a matlab (.mat) data file, \map{} will
    read frames {\tt num1} to {\tt num2} from the file.
    
  \item[{\tt-i increment}] = difference between each frame number.
    With the last few versions, this would still read in all the frames, but
    this version acts more like the versions prior to that, and subsamples
    the data.  
    
  \item[{\tt-ph maxpotval}] = maximum data value in ``user'' scaling
    mode.  This sets one option for setting the range used in 
    scaling the data value to colours and contours.  You can select other
    ranges from the menu and can select this one again with
    Scaling->Range->Command-line specified range. 
    
  \item[{\tt-pl minpotval}]
    = minimum data value in ``user'' scaling mode. 
    
  \item[{\tt-cs contour-spacing}] = spacing between contours set by the
    user.  This provides a menu option for selecting contours by setting a
    constant spacing rather than deriving the spacing from the desired
    number of contours and the range of data values.  Note that the spacing
    will not always be a the command-line set value - \map{} will divide
    the range by the specified value and set the number of contours as that
    number, and then determine the contour values by using that number of
    contours with the currently- selected scaling function.  You can select
    other numbers of contours from the menu and can select this again with
    Contours->Number of Contours->Command-line spacing
    
  \item[{\tt-ps scaleval}] = scaling value by which \map{} multiplies each
    potential value as it reads from the file(s).  This option tries to
    make use of any unit information available in a time series data file
    and alters the unit value available to \map{} for display.  The
    resulting scaling of the data is permanent for the current instance of
    \map{}.
        
  \item[{\tt -ch channels-filename}] = file in {\em channels\/} format
    containing an entry for each node in the geometry which points to the
    associated location in the data array.  The value of this pointer is
    also the number that is written next to node locations when channel
    numbers are displayed.  See section~\ref{sec:leadfiles} for more
    information on the channels file format.
        
  \item[{\tt -lm landmarks-filename}] = file in {\em landmarks\/} format
    containing a set of landmark segments, divided into categories.  Each
    category has a word depicting the landmark type.  Each lines within the
    categories contains three points (x,y,z) and an associated radius,
    which may have a different effect based on the type of landmark.
    See section~\ref{sec:landmarkgeom}.

%    \item[{\tt -ch channel-links-filename}] = file in {\em channel links\/}
%          format containing an entry for each node in the geometry which
%          points to the associated location in the data array and to the
%          string array for display at that node.  See
%          section~\ref{sec:leadfiles} for more information on the 
%          channel links file format.
  
  \item[{\tt-ff fidfile}] = Ascii file containing fiducial information.
  Information may be specified for each node for an arbitrary set of 
  fiducial data.  See section~\ref{sec:fidfiles}.
      
  \item[{\tt-sl surfnum}] = surface number to which the scaling for this
    surface is to be slaved.  The idea here is to have surfaces locked in
    the way they scale and display the data; in this way, one can compare
    colors across surfaces to determine relative values of the local scalar
    data.
    
  \item[{\tt -t timesignal-lead-number}] = number of the node to be used
    for the display of a time signal in its own window.  The number refers
    to either a node number in the geometry or, if a leadlinks file is
    present, the \emph{lead} number.  This command is optionally used in
    conjunction with the -at command, to specify a node and place its
    window accordingly.  If the -at option is not present, map3d will
    choose a default window location.  Multiple invocations of this option
    are possible for each surface, providing the option to open several
    windows per surface.  At any time during the operation of the \map{}
    the user can select a new node via the pick mode menu item and have the
    time signal from that node displayed (see
    Section~\ref{sec:control-picking} for details).
    
  \item[{\tt-at xmin xmax ymin ymax}] = set the absolute location in pixels
    of a time signal window associated with the current surface.  As with
    the \texttt{-as} option, the origin is in the lower left corner of the
    screen and the full screen resolution of an SGI screen with 19-inch
    monitor typically supports 1280 by 1024 pixels.  This command is
    optionally used in conjunction with the -t command, to specify a node
    and place its window accordingly.  If the -t option is not present,
    map3d will choose a default node (the first node in the geometry).
    Multiple invocations of this option are possible for each surface,
    providing the option to open several windows per surface.
        
\end{description}

\subsection{Other usage parameters}
\map{} accepts many other command-line parameters to customize the display.
These are optimally used in a script, but the following sections showcase
some of the options not described above.  Some of these options will
only be pertinent when used from a script file (see Section~\ref{sec:scripts}).

\subsubsection{Additional global parameters}
These parameters affect all surfaces or windows in \map{}.

\begin{description}
  \item[{\tt-bw borderWidth titleHeight}] Specifies the border width and the 
    height of the title bar in pixels used in window placement calculations.
    
  \item[{-l general transformation frame}] Specifies whether the general, 
    transformation, and frame locks should be turned on.  Specify a 1 or
    a 0 for each, where 1 signifies to enable the lock, and 0 to disable.

  \item[{-pm mode}] Specifies the pick mode, where mode is an integer.
    \begin{verbatim}
    0  New Window Mode
    1  Refresh Window Mode
    2  Node Info Mode
    3  Triangle Info Mode
    4  Triangulate Mode
    5  Flip Triangle Mode
    6  Edit Node Mode
    7  Edit Landmark Mode
    8  Delete Node Mode
    \end{verbatim}
    
  \item[{-sc range function mapping}]  Describes the scaling model, with the specified 
    scaling range, function, and mapping, each being an integer.
    \begin{verbatim}
    Range (data min and max is relative to):
    0  Local
    1  All frames in current surface
    2  All surfaces in current frame
    3  All surfaces and frames
    4  All surfaces in current group in current frame
    5  All surfaces and frames in current group
    6  Slave scaling in current frame
    7  Scave scaling in all frames
    
    Function
    0  Linear
    1  Exponential
    2  Logarithmic
    3  Lab7
    4  Lab13
    
    Mapping
    0  Symmetric
    1  Separate
    2  True
    3  About Midpoint
    \end{verbatim}
  \end{description}
  
\subsubsection{Additional surface parameters}
These options affect the current surface, if placed after the first [{-f}] on the command
line, or all surfaces if placed before.
\begin{description}
  \item [{-sf size}]  Size of the small font, from 0-9
  \item [{-mf size}]  Size of the medium font, from 0-9
  \item [{-lf size}]  Size of the large font, from 0-9
  \item [{-cm map}]   Which colormap the data will use, where map is an integer:
  \begin{verbatim}
  0  Rainbow
  1  Green to Red
  2  Grayscale
  3  Jet
  \end{verbatim}
  \item [{-sm mode}] Selects the shading mode of surface data, where mode is an integer:
  \begin{verbatim}
  0  No shading
  1  Flat
  2  Gouraud
  3  Banded
  4  Textured Gouraud
  \end{verbatim}
  \item [{-rm mode}] Selects the mesh rendering mode, mode being an integer:
  \begin{verbatim}
  0  No rendering (mesh is not displayed)
  1  Points Only
  2  Mesh Elements
  3  Mesh Connectivity
  4  Elements and Connectivity
  5  Points and Connectivity
  6  Render only the mesh elements with no surface data
  \end{verbatim}
  \item[{-ic num}] Whether or not to invert the colormap, 1 meaning yes, 0 meaning no
  \item[{-nco num}] Sets the number of contours to num
  \item[{-rq w x y z}] Sets the rotation of the surface to the quaternion formed by (w,x,y,z)
  \item[{-tc x y z}] Sets the translation of the surface to (x,y,z)
  \item[{-zf vfov}] Sets the ``zoom factor'' or field of view to vfov
  \item[{-el num}] Whether or not to light the surface, 1 meaning yes, 0 meaning no
  \item[{-ef num}] Whether or not to display the surface with fog, 1 meaning yes, 0 meaning no
  \item[{-gn num}] Assigns the group number to num for group scaling
  \item[{-sco num}] Whether or not to show contours, 1 meaning yes, 0 meaning no
  \item[{-nc num}] Whether or not to show the negative contours as dashes, 1 meaning yes, 0 meaning no
  \item[{-x num}] Whether or not to show axes, 1 meaning yes, 0 meaning no
  \item[{-xc r g b}] Specifies the axes' color, with RGB numbers from 0-255
  \item[{-sit num}] Whether or not to show informational text, 1 meaning yes, 0 meaning no
  \item[{-sli num}] Whether or not to show the lock icons, 1 meaning yes, 0 meaning no
  \item[{-nma sphere mark value}] Settings for Node Marking: all.  
    Sphere indicates whether to draw spheres for all nodes, 1 meaning yes, 0 meaning no.
    Value indicates whether to color the spheres based on their data value, 1 meaning yes, 0 meaning no.
    Mark indicates what to display at the nodes, mark being an integer:
    \begin{verbatim}
    0  nothing
    1  Node number
    2  Channel number
    3  Data value
    \end{verbatim}
  \item[{-nme sphere mark}] Settings for Node Marking: extrema, where sphere and mark 
  have the same meanings as [{-nma}], except the affect the extrema nodes
  \item[{-nmp sphere mark}] Settings for Node Marking: pick, where sphere and mark 
  have the same meanings as [{-nma}], except the affect only the nodes that have an active ``Time Series''
  Window open.
  \item[{-nml sphere mark}] Settings for Node Marking: leadlinks, where sphere and mark 
  have the same meanings as [{-nma}], except the affect only the nodes that are
  represented in the active leadlinks file, except that a mark of 3 will display the lead label
  instead of the data value.
  \item[{-Sll value}] Sets the size of the leadlinks points, from 0-9.
  \item[{-Cll r g b}] Sets the color of the leadlinks points, with RGB numbers from 0-255
  \item[{-rf num}] Whether or not to draw fiducials, 1 meaning yes, 0 meaning no.  
  
  
\end{description}

%%%%%%%%%%%%%%%%%%%%%%%%%%%%%%%%%%%%%%%%%%%%%%%%%%%%%%%%%%%%%%%%%%%%


%%% Local Variables: 
%%% mode: latex
%%% TeX-master: "manual"
%%% End: 
