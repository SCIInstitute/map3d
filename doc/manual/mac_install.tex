% -*-latex-*-
% Document name: mac_install.tex
% Creator: J.R. Blackham [blackham@sci.utah.edu]
% Last update: Fri Jun 30 15:00:00 2004 by J.R. Blackham
%    - release 6.2
%%%%%%%%%%%%%%%%%%%%%%%%%%%%%%%%%%%%%%%%%%%%%%%%%%%%%%%%%%%%%%%%%%%%%%
\subsection{Mac OS X Installation Instructions}
\label{sec:mac-install}

The Mac OS X installation is relatively straightforward.  You'll need to 
download \map{}'s dependencies, then download \map{} itself.

\subsubsection{Mac OS X Dependencies}

Map3d requires the gtk+2 and gtkglext libraries.  You can easily
install these libraries using fink.

If you do not currently have fink installed on your system you will
need to go to \htmladdnormallink{http://fink.sourceforge.net}{http://fink.sourceforge.net} and
follow the instructions on how to install it and the gtk+2 and
gtkglext libraries.

\subsubsection{Mac OS X Executable}

Download the \texttt{map3d-6.4-mac.tar.gz} file from the \map{}
\htmladdnormallink{\texttt{download page}}
{http://www.sci.utah.edu/software/map3d.html}
and unzip it to a directory of your choice.  We will call that
\texttt{RUN-DIR}. This is the directory from which you will run \map{}.





%%% Local Variables: 
%%% mode: latex
%%% TeX-master: "manual"
%%% End: 
