% -*-latex-*-
% Document name: examples.tex
% Creator: Rob MacLeod [macleod@vissgi.cvrti.utah.edu]
% Last update: September 4, 2000 by Rob MacLeod
%    - created from old version of the manual
%%%%%%%%%%%%%%%%%%%%%%%%%%%%%%%%%%%%%%%%%%%%%%%%%%%%%%%%%%%%%%%%%%%%%%
\section{Examples}

\subsection{Demos}

The easiest way to use \map{} for complicated sets of surfaces and/or
potentials files is to set up a script file.  The composition of script
files is described in Section~\ref{sec:controlscripts} in more detail.
Examples of such files are given in the subdirectory demos.  Any of these
can be executed simply by typing the filename in a window.  You may want to
start from these files and alter them to experiment with other features of
\map{}.  Examples of data files, both geometry and potential data, are to
be found in the demos subdirectories.


\subsection{Examples}

\subsubsection{Data and geometry filepaths}

Here is a set of command line examples that all run the same data, and show
some of the variability that is possible with \map{}.  All three examples
access the geometry in the file {\tt ~mapdata/bt25mar92.geom} and the data
in the file {\tt bt25mar92-1.tsdf}, which we assume are stored in ~mapdata
(home directory for the data account for Bruno Taccardi's group).

\noindent
Version 1: 

\begin{verbatim}
map3d  -f ~mapdata/bt25mar92.geom \
       -dp /vax/odisk0/bt25mar92pack/ \
       -p ~mapdata/bt25mar92-1.tsdf
\end{verbatim}

Note the use of \verb=~mapdata= to specify the home directory of
the user (mapdata in this case), a good way to achieve some independence of
which computer is actually running the script.  

The data file for this example is assumed to be on the optical disk \#0 an
is set by the {\tt -dp} option. 

In this example, the geometry file is explicitly specified, which requires
that we know the name of the geometry file for the particular .tsdf file.
It turns out that the .tsdf file knows this already for us.  But the
geometry filename specified in the .tsdf file has no directory and if the
script is not run from mapdata's directory, \map{} will not find the geometry
file.  We need a way to tell \map{} that the geometry file is to be found
elsewhere.  We can state this explicitly, using the method above, or more
generally, as shown in the next version.

\noindent
Version 2:

\begin{verbatim}
map3d  -dp /vax/odisk0/bt25mar92pack/ \
	-gp ~mapdata/ \
	-f ~mapdata/bt25mar92-1.tsdf
\end{verbatim}

Here we use the {\tt -gp ~mapdata} line to tell \map{} that any geometry
file(s) listed in a .tsdf file are to be found in mapdata's home directory.

The other thing to note in both these examples is that the directory in
which the data files are to be found has been redirected to {\tt
vax/odisk0/bt25mar92pack}.  This does not mean that the .tsdf file is
located here (the -p line in version 1 and the -p line in version 2
indicate where the .tsdf file is located) but tells \map{} that the
directory where the .pak files that contain the actual numerical data is
not the same as the one specified for the .tsdf file.  Instead, the .pak
files can be found in {\tt /vax/odisk0/bt26mar92pack/}, an optical disk on
the Vax.  The purpose of this is to gain flexibility in where the
individual .pak (or .raw) files are located.  You do not have to move files
from the optical disk to the disk, for example, in order to read them with
\map{}.

A third version of the script does the same as version 2.

\noindent
Version 3:

\begin{verbatim}
map3d  -f ~mapdata/bt25mar92-1.tsdf \
	-dp /vax/odisk0/bt25mar92pack/ \
	-gp ~mapdata/
\end{verbatim}

The difference here is that the {\tt -dp} and {\tt -gp} arguments can be
located either before the first {\tt -f filename}, in which case they apply
globally to all subsequent surfaces, or after the {\tt -f filename} to
which they should apply.

Another, even simpler form of the script could be used if we first specify
an environmental variable to direct the search for .geom and .pak files.

\noindent
Version 4:

To have this one work, you must have environmental variables specified as

\begin{verbatim}
eli:macleod> echo $MAP3D_DATAPATH
/vax/odisk0/bt25mar92pack/
eli:macleod> echo $MAP3D_GEOMPATH
/vis/u/mapdata/
\end{verbatim}

With these both set, then the \map{} command becomes even simpler.

\begin{verbatim}
${HOME}/gl/cj/map3d -f ~mapdata/bt25mar92-1.tsdf 
\end{verbatim}

These environmental variables can be set from within a script, or manually,
and the method used depends on the shell you are using for Unix.  The most
common shell for the CVRTI is the Korn shell (ksh) and the method used here
would be

\begin{verbatim}
     MAP3D_DATAPATH=/vax/odisk0/bt25mar92pack/
     export MAP3D_DATAPATH
     MAP3D_GEOMPATH=/vis/u/mapdata/
     export MAP3D_GEOMPATH
\end{verbatim}
%
or for the C-shell (if you insist)
%
\begin{verbatim}
         setenv MAP3D_DATAPATH /vax/odisk0/bt25mar92pack
\end{verbatim}

A further way to achieve the same end would be to define a defaults file
(see .map3drc file in section~\ref{sec:defaults}) with the lines

\begin{verbatim}
geomfile_path = /vis/u/mapdata/
datafile_path = /vax/odisk0/bt25mar93pack/
\end{verbatim}

And still another would be to use Phil's odisk program to interactively
select the location of the .pak/.raw files.  Just enter ``odisk'' from the
console of either SGI workstation and select the optical disk, directory,
and a single file from the list.

So, as always, there is more than one way to skin the cat, and you need
only select the one that seems simplest to you.

\subsection{Data Directories}

The following directories contain data files (geometries and potentials)
which can be used to play with \map{}.  They are all subdirectores of
~macleod/gl/map3d and are read-accessible by anyone, and may even contain some
bits of documentation as to what they are (but don't count on it).

\begin{description}
  \item[cube/] = simulations using a dipole source and cubic volume
        conductor with anisotropic inhomogeneities
  \item[dal/] = data from the Dalhousie torso model with some
        potentials
  \item[spheres/] = artificial data of potential distributions
        over multiple, triangulated spherical sections
        
\end{description}

\subsection{Important files}

\begin{description}
  \item[/usr/local/bin/map3d] a symbolic link to a script, which in turn
        calls the appropriate executable of \map{}
  \item [/usr/local/bin/map3d-elan] executable version of \map{} for elan
        level graphics 
  \item [/usr/local/bin/map3d-vgx] executable version of \map{} for vgx
        level graphics 
  \item [/usr/local/doc/map3d-manual.dvips] postscript file of the map3d
        documentation 
  \item [www/cvrti/utah.edu/\~{}macleod/docs/map3d] HTML version of the map3d
        documentation 
\end{description}


\subsection{Script file examples}
\label{sec:examplescripts} 

Below are some sample scripts, from simple, to fairly complex:

\paragraph{Set the geometry, data, and window size and location}

\begin{verbatim}
         map3d -f ${HOME}/torso/geom/dal/daltorso.fac \
               -as 100 500 300 700 \
               -p ${HOME}/maprodxn/andy3/10feb95/data/cooling.tsdf \
               -s 1 1000
\end{verbatim}

\paragraph{Set some environment variables, then layout the whole display}

\begin{verbatim}
#!/bin/sh
# A script for the spmag 1996 article
#
######################################################################
map3d=/usr/local/bin/map3d
map3d=${ROBHOME}/gl/map3d/map3d.sh
MAP3D_DATAPATH=/vax/odisk0/bt26mar91pack/
export MAP3D_DATAPATH
echo "MAP3D_DATAPATH = $MAP3D_DATAPATH"
basedir=/u/macleod/maprodxn/plaque/26mar91
$map3d -b -nw \
	-f $basedir/geom/525sock.geom \
	-as 150 475 611 935 \
	-at 150 475 485 610 -t 237 \
	-p $basedir/data/pace-center.tsdf@1 \
	-s 65 380 \
	-f $basedir/geom/525sock.geom \
	-as 476 800 611 935 \
	-at 476 800 485 610 -t 237 \
	-p $basedir/data/pace-center.tsdf@1 \
	-s 65 380 \
	-f $basedir/geom/525sock.geom \
	-as 150 475 176 500 \
	-at 150 475 50 175 -t 237 \
	-p $basedir/data/pace-center.tsdf@1 \
	-s 65 380 \
	-f $basedir/geom/525sock.geom \
	-as  476 800 176 500 \
	-at  476 800 50 175 -t 237 \
	-p $basedir/data/pace-center.tsdf@1 \
	-s 65 380 
\end{verbatim}

\paragraph{A script with command arguments and built in logic}

If you can figure this one out, you are a shell script guru!  Running it is
child's play and it displays epicardial and tank surface potentials for two
different runs from any Andy2 tank experiment.  The beats are time aligned
based on the values in a separate parameter file that must exist for the
particular experiment.

\begin{verbatim}
#!/bin/sh
# Filename: map3d-comp2.sh
# Display epicardial and torso potentials from two runs of the same file
# using a standard layout and a file or parameters to feed the script
# with all it needs to know.
# The minimum argument of this script is the name of the parameter script
# as without it, we have no idea what to do.
# The parameters are stored in a file, that when executed, sets them
# Format is therefore very flexible.
#
# Assumptions:
#    That geometry is in ${date}_sock.pts and .fac
#    That landmarks are in ${data}_cor.lmarks
#    That we have 64 lead socks with the standard channels (193--256)
#    That the 192 lead torso geometry is used.
#    
#
# Last update: Sun Jun  2 18:12:42 1996
#    - fixed a few things in the run number entry
# Last update: Wed Dec 13 12:28:46 1995
#    - still trying to expand and generalize
# Last update: Tue Oct 24 10:22:10 1995
#    - created
######################################################################
#### Set up parameters ####

MAP3D=${HOME}/gl/cj/map3d
MAP3D=/usr/local/bin/map3d
MAPRODXN=${HOME}/maprodxn
USAGE="\nYou can run this script with arguments as follows:\n\
     $0 paramfile.m3dp [-ocsd0145etc] n1 n2, where\n\n\
     paramfile.m3dp is a script that sets the local parameters\n\
     -o = optical disk\n\
         0=odisk0; 1-odisk1; 4-odisk4; 5-odisk5\n\
     -b = backup disk\n\
     -s = scratch disk\n\
     -d = data disk\n\
     -e = show epi only \n\
     -t = show torso only \n\
     -c = show coronaries\n\n\
     -v = show V1-V6 on tank\n\n\
     -i = show integral maps\n\
     n1 and n2 are the first and second run numbers\n\n\
     Make sure to GROUP ALL ARGUMENTS into one block!!! \n\n\
        e.g.,  $0 -co1 2 5\n\
     to get data from odisk1 and use runs 2 and 5 and show coronaries\n"

##### Check arguments #####
if [ $# -lt 1 ]
  then
  echo "\n *** Error in entry "
  echo " *** We need at least the name of the .m3dp (parameter) file ***"
  echo "$USAGE"
  exit
else
  paramfile=$1
  shift
fi

##### Execute script to set local params #####
. $paramfile

##### Check other input #####
run1=""
run2=""
#echo "Arguments are  $*"
doshift=0
showcor=""
showvleads=""
epionly=""
torsoonly=""
showint=""

if [ $# -gt 0 ]
  then
  odisknum=""
  dataloc=""
  while getopts bceiostvd0145 optval
    do 
    case $optval in
      o | d | s | b) dataloc=$optval
                 doshift=1;;
      c)         showcor=1
                 doshift=1;;
      e)         epionly=1
                 doshift=1;;
      i)         showint=1
                 doshift=1;;
      v)         showvleads=1
                 doshift=1;;
      t)         torsoonly=1
                 doshift=1;;
      0 | 1 | 4 | 5)  odisknum=$optval
		 doshift=1;;
      \?)        echo $USAGE
                exit 2;;
    esac
  done
#  echo "Arguments after getopts are $*"
  if [ $doshift ]
    then
    shift
#    echo "Do the shift"
  fi
  # Read in the run numbers for first and second runs
  if [ -n "$1" ]
    then
    run1=$1
    shift
  fi
  if [ -n "$1" ]
    then
    run2=$1
    shift
  fi
  
#  echo "Arguments are $*"
#  echo "dataloc = $dataloc"
#  echo "odisknum = $odisknum"
  if [ "$dataloc" = o ]
    then
#    echo "Check odisk value = $odisknum"
    if [ "$odisknum" != 0 -a "$odisknum" != 1 -a "$odisknum" != 4\
      -a "$odisknum" != 5 ]
      then
      echo "+++ Odisk number was $odisknum so we set it to 0"
      odisknum=0
    fi
  fi
else
  echo "$USAGE"
  echo "\n We need some more input information"
  echo "\n If we need to know where to look for Vax data files"
  echo " where should we go?"
  echo " Enter the location of the Vax data files"
  echo " Optical disk drive 0 ................. 0"
  echo " Optical disk drive 1 ................. 1"
  echo " Optical disk drive 4 ................. 4"
  echo " Optical disk drive 5 ................. 5"
  echo " Datadisk ............................. d"
  echo " Scratchdisk .......................... s"
  echo " Backupdisk ........................... b"
  echo " $MAP3D_DATAPATH ............... return"
  echo "   Your entry? \c"
  read inchar
  case "$inchar"
    in
    0) dataloc=o
    odisknum=0;;
    1) dataloc=o
    odisknum=1;;
    d) dataloc=d;;
    s) dataloc=s;;
    b) dataloc=b;;
  esac
fi

###### Check fore silly combos #####

if [ "$epionly" -a "$torsoonly" ]
  then
  echo "*** Entry error: You cannot use -e and -t or there is nothing to see"
  exit 1
fi

###### Get the run numbers manually #####

needruns=""

if [ -z "$run1" -o -z "$run2" ]
  then
  needruns="t"
elif [ "$numruns" -gt 0 ]
  then
  if [ "$run1" -gt "$numruns" -o "$run2" -gt "$numruns" ]
    then
    needrun="t"
  fi
fi
if [ -n "$needruns" ]
  then
  echo " Now enter the first and second run numbers"
  echo " ? \c"
  read run1 run2
fi
if [ "$run1" -le 0 ]
  then
  run1=1
fi
if [ "$run2" -le 0 -o "$run2" -le "$run1" ]
  then
  run2=`expr $run1 + 1`
fi
if [ -n "$numruns" -a "$run2" -gt "$numruns" ]
  then
  run2="$numruns"
fi
#echo "run1 = $run1 and run2 = $run2"

###### Now see about the optical disk drive and set up some variables. #####

if [ $dataloc = o ]
  then
  echo " Make sure the optical disk called RMSPACK-1A is in drive $odisknum"
  MAP3D_DATAPATH=/vax/odisk${odisknum}/${thename}${thedate}pack/
elif [ $dataloc = s ]
  then
  MAP3D_DATAPATH=/vax/scratch/${datadir}/${thename}${thedate}/
elif [ $dataloc = b ]
  then
  MAP3D_DATAPATH=/vax/backup/${datadir}/${thename}${thedate}/
elif [ $dataloc = d ]
  then
  MAP3D_DATAPATH=/vax/data/mapping/
fi
export MAP3D_DATAPATH
echo "\n MAP3D_DATAPATH exported as $MAP3D_DATAPATH\n"

if [ ! -n "$ROBHOME" ]
then
   ROBHOME=/u/macleod
   export ROBHOME
fi

###### DO we have cor file? #####

if [ "$showcor" -a -z "$torsoonly" ]
  then
  corfile="       -lm ${MAPRODXN}/andy3/${thedate}/geom/${thedate}_sock.lmarks"
else
  corfile=""
fi

###### DO we have vleads leadlinks file? #####

if [ "$showvleads" -a -z "$epionly" ]
  then
  vleadsfile="       -ll ${ROBHOME}/torso/geom/andy3/leadset1_v1-6.leadlinks"
else
  vleadsfile=""
fi

###### Set up the parameters fore running the program #####

# PArse datafilename four .tsdf extension, and strip it
if echo $datafile | grep .tsdf > /dev/null
  then
  datafile=`echo $datafile | sed 's/.tsdf//'`
  echo "Value of datafile updated to $datafile\n"
fi

if [ -n "$showint" ]
  then
  datafile=${datafile}_int
  echo "Value of datafile updated to $datafile\n"
fi

cd ${MAPRODXN}/andy3/${thedate}
#if [ -z "$numruns" ]
#  then
#  numruns=7
#fi
#runnum=1
# while [ $runnum  -le $numruns ]
#   do
#   if [ $run1 = $runnum ]
#     then
#     echo "Match fore run1 = $run1"
#     start1=${onset}${runnum}
#     echo "Start1 = $start1"
#     end1=`expr $start1 + $duration`
#   fi
#   if [ $run2 = $runnum ]
#     then
#     start2=${onset}${runnum}
#     end2=`expr $start2 + $duration`
#   fi
#   runnum=`expr $runnum + 1`
# done

########### Set up the start and end of the frames numbers
# This is no elegant, and must be expanded when numframes climbs
# but is appears to work

case "$run1"
  in 
  1) start1=$onset1;;
  2) start1=$onset2;;
  3) start1=$onset3;;
  4) start1=$onset4;;
  5) start1=$onset5;;
  6) start1=$onset6;;
  7) start1=$onset7;;
  8) start1=$onset8;;
  9) start1=$onset9;;
 10) start1=$onset10;;
 11) start1=$onset11;;
 12) start1=$onset12;;
 13) start1=$onset13;;
 14) start1=$onset14;;
 15) start1=$onset15;;
 16) start1=$onset16;;
esac
end1=`expr $start1 + $duration`
case "$run2"
  in 
  1) start2=$onset1;;
  2) start2=$onset2;;
  3) start2=$onset3;;
  4) start2=$onset4;;
  5) start2=$onset5;;
  6) start2=$onset6;;
  7) start2=$onset7;;
  8) start2=$onset8;;
  9) start2=$onset9;;
 10) start2=$onset10;;
 11) start2=$onset11;;
 12) start2=$onset12;;
 13) start2=$onset13;;
 14) start2=$onset14;;
 15) start2=$onset15;;
 16) start2=$onset16;;
esac
end2=`expr $start2 + $duration`

if [ -n "$showint" ]
  then
  start1=1
  end1=4
  start2=1
  end2=4
fi

if [ -z "$start1" ]
  then
  echo "Start1 never set so exit"
  exit 1
elif [ -z "$start2" ]
  then
  echo "Start2 never set so exit"
  exit 2
fi

##### Now, set up the run string for the progr #####
  
epicommand="\n\
        -f ${MAPRODXN}/andy3/${thedate}/geom/${sockgeom}.fac \n\
	-as 40 440 551 950 \n\
	-t ${epilead} \n\
	-at 840 1240 825 950 \n $corfile \n\
	-p data/${datafile}.tsdf@${run1} -s $start1 $end1 \n\
	-cl ${ROBHOME}/torso/geom/socks/sock64/sock64.channels \n\
	-sl 2 \n\
	-f ${MAPRODXN}/andy3/${thedate}/geom/${sockgeom}.fac \n\
	-as 441 840 551 950 \n\
	-t ${epilead} \n\
	-at 840 1240 700 825 \n $corfile \n\
	-p data/${datafile}.tsdf@${run2} -s $start2 $end2 \n\
	-cl ${ROBHOME}/torso/geom/socks/sock64/sock64.channels \n\
	-sl 2"

torsocommand="\n\
	-f ${ROBHOME}/torso/geom/andy3/leadset1.fac \n\
	-as 40 440 150 550 \n\
	-t ${torsolead} \n\
	-at 840 1240 400 525 \n $vleadsfile \n\
	-p data/${datafile}.tsdf@${run1} -s $start1 $end1 \n\
	-cl ${ROBHOME}/torso/geom/andy3/leadset1.channels \n\
	-sl 4 \n\
	-f ${ROBHOME}/torso/geom/andy3/leadset1.fac \n\
	-as 441 840 150 550 \n\
	-t ${torsolead} \n\
	-at 840 1240 275 400 \n $vleadsfile \n\
	-p data/${datafile}.tsdf@${run2} -s $start2 $end2 \n\
	-cl ${ROBHOME}/torso/geom/andy3/leadset1.channels \n\
	-sl 4 "

if [ "$epionly" ]
  then
  command="${MAP3D} -nw -b $epicommand"
elif [ "$torsoonly" ]
  then
  command="${MAP3D} -nw -b $torsocommand"
else
  command="${MAP3D} -nw -b $epicommand $torsocommand"
fi

##### And, finally, run the program #####
echo "Run program as $command"
$command
\end{verbatim}


