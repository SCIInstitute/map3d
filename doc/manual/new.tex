% -*-latex-*-
% Document name: new.tex
%
% What's new in the map3d work?
%
% Last update: Sun Jan 21 10:05:20 2001 by Rob MacLeod
%    - created
%%%%%%%%%%%%%%%%%%%%%%%%%%%%%%%%%%%%%%%%%%%%%%%%%%%%%%%%%%%%%%%%%%%%%%
\section{What's New?}

Here we list new features and changes relevant to the latest version of
\map{}.  New items will appear here first and then move to the appropriate
sections of the manual

\subsection{Bug Fixes}

This new release contains virtually all bug fixes.  As with any first beta
release, there were things that did not work, things that only partially
worked and things we did not want to work!  Here are some samples:
%
\begin{description}
  \item [Reading .pot files: ] fixed the code that reads the ASCII .pot
        files so that it actually works.
  \item [Fonts: ] the search for decent font rendering within GLUT goes on
        and we tried some changes, with only moderate success.  If anyone
        has some ideas, please let us know.
  \item [LinucPPC support: ] for those millions and millions of LinuxPPC
        fans, you can now run \map{}.  For the rest of you who have no idea
        what LinuxPPC is, this is a port of Linux to the PowerPC, which is
        what makes Macintosh computers go.  I used a pretty simple LinuxPPC
        configuration on a year-old kernel so am hoping it will work for you.
\end{description}



