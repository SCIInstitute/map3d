% -*-latex-*-
% Document name: intro.tex
% Creator: Rob MacLeod [macleod@vissgi.cvrti.utah.edu]
% Last update: September 2, 2000 by Rob MacLeod
%    - created
% Last update: Sun Sep 24 16:41:08 2000 by Rob MacLeod
%    - Version 5.0Beta release edition
% Last update: Mon Jul 23 13:28:30 2001 by Rob MacLeod
%    - release 5.2
% Last update: Fri Mar 1 20:00:00 2002 by Bryan Worthen
%    - release 5.3
% Last update: Sun Feb  2 01:54:43 2003 by Rob MacLeod
%    - Release 5.4
% Last update: Thu May 19 12:35:15 2005 by Rob Macleod
%    - for version 6.3
g%%%%%%%%%%%%%%%%%%%%%%%%%%%%%%%%%%%%%%%%%%%%%%%%%%%%%%%%%%%%%%%%%%%%%%
\section{Introduction}

This document describes the function and usage of version\~version{} of the
program \map{}, a scientific visualization application originally developed
at the Nora Eccles Harrison Cardiovascular Research and Training
\htmladdnormallink{(CVRTI)}{http://www.cvrti.utah.edu} and now under
continued development and maintenance at the Scientific Computing and
Imaging Institute \htmladdnormallink{(SCI)}{http://www.sci.utah.edu} at the
University of Utah.  The original purpose of the program was to
interactively view scalar fields of electric potentials from measurements
and simulations in cardiac electrophysiology.  Its present utility is much
broader but continues to focus on viewing three-dimensional distributions
of scalar values associated with an underlying geometry consisting of node
points joined into surface or volume meshes.

\map{} has been the topic of some papers
\cite{RSM:Mac92c,RSM:Mac92d,RSM:Mac93,RSM:Mac93a} and a technical report
\cite{RSM:Mac94d} and we'd love it if you would reference at least one of
them (perhaps \cite{RSM:Mac93} or \cite{RSM:Mac93a} are the easiest ones to
get copies of) as well as this manual when you publish results using it.
There have been many many more papers that use \map{} and the list keeps
growing.\cite{RSM:Mac93,RSM:Mac93a,RSM:Mac94d,RSM:Mac94c,RSM:Mac94f,RSM:Mac95d,RSM:Mac97,RSM:Mac98,RSM:Mac2000a,RSM:Mac2001,RSM:Pun98,RSM:Pun99,RSM:Tac92,RSM:Tac92b,RSM:Tac94,RSM:Tac96,RSM:Tac96b,RSM:Tac97b,RSM:Ni98,RSM:Ni99,RSM:Ni2000b,RSM:Lux96,RSM:Lux2001,RSM:Joh93b,RSM:Joh94,RSM:Joh94b,RSM:Pun2003,RSM:Ser2002}

One of the big changes in version~\version{} is that we are now completely
open source.  People can download not only the executable but also the
complete source code for the program.  Please note that we do not have a
good way yet to incorporate changes people outside our little group make
to the program.  If you do wish to change and then contribute back, please
let us know as soon as possible and we can try and coordinate as best we
can.  Of obvious interest is when someone ports \map{} to another
platform---please let us know about this and we can add it to the list and
release it with the rest.

\subsection{Acknowledgments}

The history of \map{} goes back to 1990 and the first few hundred lines of
code were the product of a few hours work by Mike Matheson, an inspired
visualization specialist, now with SGI in Salt Lake City.  This was my
introduction to GL and C and this program became my personal sand box to
play in.  Along the way, Phil Ershler made valuable contributions in
figuring out the magic of Formslib for some user interface controls and
developing with me \emph{graphicsio}, the geometry and data file library
that supports \map{}.  Ted Dustman has recently taken up maintenance and
extensions of \emph{graphicsio} and remains my main man when I need
programming lessons.

This is one in a series of ``new'' versions of \map{}, the series (labeled
5.x or above) that marks the move from GL to OpenGL library and thus to
becoming truly portable.  In fact, we call the old one \mapgl{} now to
indicate its links to SGI's original GL library.  We seem permanently stuck
in the middle of this big conversion project, moving support to OpenGL and
adding lots of power as we convert functionality.  The reason for the
version 6.x, was the move to gtk as the GUI library with which we create
all the dialog and display elements of the program.  This move has allowed
us to extend dramatically the set of dialog boxes \map{} offers and this
newest version~\version{} contains many examples.

There are some people who have been instrumental in the process and deserve
special mention.  Chris Moulding is a graphics programmer and general
software whiz who surveyed my sand box architecture, pulled together the
essential walls, created new ways to make rooms, and still left lots of the
sand box around so we could continue to play.  From version 5.2 onward,
Bryan Worthen replaced Chris and really has found the spirit of \map{}.
Bryan has become the main driving force behind the actual work of coding
and fixing.  He strayed off to some other project for a while, but never
lost his love for \map{}; we are really pleased that he has returned to
pick up the torch again.  Most recently, J.R. Blackham has joined the team
while still an undergraduate in Computer Science at Utah.  Jeroen Stinstra
is my super-postdoc, helpful in more ways than I knew I even needed and
full of inventive ideas.  He has created the support for MATLAB that we use
in \map{} (and the SCIRun project) and is best bug-catcher I know.

The largest thanks must go to the users of \map{}, who provided the real
inspiration and identified the needs and opportunities of such a program.
Among the most supportive and helpful are Bruno Taccardi, Bonnie Punske,
and Bob Lux, all colleagues of mine at the CVRTI. Dana Brooks and his
students from Northeastern University are also regular users who have
provided many suggestions and great enthusiasm.  Also invaluable in the
constant improvement of the program are my post docs, Jeroen Stinstra, and
graduate student Quan Ni, Rich Kuenzler, Bulent Yilmaz, Bruce Hopenfeld,
Shibaji Shome, Lucas Lorenzo, Andrew Shafer, and Zoar Englemann.  They give
me new energy every day and remind me why I am a professor.  Notable new
additions to the family are Randy Thomas from Universite d'Evry Val
d'Essonne in Evry, France.  The great thing about Randy is that he used
\map{} to visualize concentrations of ions in his simulation of the
nephron!  Also, Ed Ciaccio from Columbia University has become a big user
and even takes it to his classes.

The first user and long-time collaborator and friend was Chris Johnson and
this new version of \map{} is possible because of the success he and I have
had in creating the SCI Institute and specifically the NIH/NCRR Center for
Geometric Modeling, Simulation, and Visualization in Bioelectric Field
Problems (\htmladdnormallink{www.sci.utah.edu/ncrr}
{www.sci.utah.edu/ncrr}).

We gratefully acknowledge the financial support that has come from the NIH,
National Center for Research Resources
\htmladdnormallink{(NCRR)}{http://www.ncrr.nih.gov} the Nora Eccles
Treadwell Foundation, and the University of Utah, which provides us with
space and materials to create this sand box.  The Nora Eccles Treadwell
Foundation has also provided support for the development of \map{} and the
huge pile of data we have used it to analyze.

\bigskip
\noindent
Rob MacLeod, May 19, 2005.


\subsubsection{Open Source License}
\label{sec:source_license}

The terms of the license agreement under which we release \map{} are simple
and as follows:

\begin{quotation}

Permission is hereby granted, free of charge, to any person obtaining a
copy of this software and associated documentation files (the
``Software''), to deal in the Software without restriction, including
without limitation the rights to use, copy, modify, merge, publish,
distribute, sublicense, and/or sell copies of the Software, and to permit
persons to whom the Software is furnished to do so, subject to the
following conditions:
\begin{enumerate}
  \item The above copyright notice and this permission notice shall be
    included in all copies or substantial portions of the Software.
  \item Use of this software in preparing any publication material must be
    cited as follows:
    
    \vspace{0.1in}
    
    R.S. MacLeod and C.R. Johnson. Map3d: Interactive scientific
    visualization for bioengineering data. In IEEE Engineering in
    Medicine and Biology Society 15th Annual International Conference,
    pages 30-31, IEEE Press, 1993.
\end{enumerate}
\vspace{0.1in}

THE SOFTWARE IS PROVIDED ``AS IS'', WITHOUT WARRANTY OF ANY KIND, EXPRESS
OR IMPLIED, INCLUDING BUT NOT LIMITED TO THE WARRANTIES OF MERCHANTABILITY,
FITNESS FOR A PARTICULAR PURPOSE AND NONINFRINGEMENT. IN NO EVENT SHALL THE
AUTHORS OR COPYRIGHT HOLDERS BE LIABLE FOR ANY CLAIM, DAMAGES OR OTHER
LIABILITY, WHETHER IN AN ACTION OF CONTRACT, TORT OR OTHERWISE, ARISING
FROM, OUT OF OR IN CONNECTION WITH THE SOFTWARE OR THE USE OR OTHER
DEALINGS IN THE SOFTWARE.
\end{quotation}


\subsubsection{Libraries used by \map{}}

\map{} incorporates the functionality of several external libraries.  They
are:
\begin{itemize}
  \item \htmladdnormallink{GTK}{http://www.gtk.org} - The GIMP Toolkit -
    Copyright (C) 1995-1997 Peter Mattis, Spencer Kimball and Josh
    MacDonald 
  \item \htmladdnormallink{GtkGLExt}{http://gtkglext.sourceforge.net/} -
    GtkGLExt - OpenGL Extension to GTK+ Copyright (C) 2002-2004  Naofumi
    Yasufuku 
  \item \htmladdnormallink{PNG}{http://www.libpng.org} - Copyright (c)
    1998-2002 Glenn Randers-Pehrson 
  \item \htmladdnormallink{Jpeglib}{http://www.ijg.org} - Copyright (C)
    1991-1998, Thomas G. Lane.
\end{itemize}

We use GTK and GtkGLExt to interface with the window manager to give us
windows with OpenGL capability, as well as giving us widgets we need for
interactive control.  We use PNG and JpegLib to be able to save
\texttt{.png} and \texttt{.jpg} images of \map{}.
All four of these libraries are covered by the \htmladdnormallink{GNU
LGPL}{http://www.gnu.org/licenses/lgpl.html}, which is included in the
distribution of \map{}.

As of version~\version{}, we also release internal libraries under the
same license as above for the rest of \map{}.  



%%% Local Variables: 
%%% Local Variables: 
%%% mode: latex
%%% TeX-master: "manual"
%%% TeX-master: "manual"
%%% TeX-master: "manual"
%%% End: 
