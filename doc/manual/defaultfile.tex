% -*-latex-*-
% Document name: defaultfile.tex
% Creator: Rob MacLeod [macleod@cvrti.utah.edu]
% Last update: September 4, 2000 by Rob MacLeod
%    - created but not included in version 1.0 of the manual
%%%%%%%%%%%%%%%%%%%%%%%%%%%%%%%%%%%%%%%%%%%%%%%%%%%%%%%%%%%%%%%%%%%%%%
\subsection{Default settings files}
\label{sec:defaults} 

\map{} looks for files containing default settings for
many parameters that are relevant to the control of the program.
The order of precedence is as follows:
%
\begin{enumerate}
  \item The {\tt-df filename} option defines the file with highest
        precedence default settings.
  \item A file named {\tt .map3drc} in the current directory (the one from
        which the application was launched) is used next.
  \item If no {\tt .map3drc} files is found in the current directory, it is
        searched for in the user's home directory (see the HOME
        environmental variable).  This file has the lowest precedence and
        is only used of the other two options are not found.
  \item The program \map{} has a set of internal defaults which are used
        if there are no external default files found.
\end{enumerate}

\newpage
The format of the default file is as follows:
%
\begin{verbatim}
           # comment line (ignored by map3d)
           parameter = value
\end{verbatim}

\noindent
where the parameters and values are taken from the following list:

\begin{center}
\begin{tabular}{|l|l|p{3.2in}|} \hline
\multicolumn{1}{|c|}{Parameter} &
\multicolumn{1}{|c|}{Values} &
\multicolumn{1}{|c|}{Meaning} \\ \hline
shadingmodel & GOURAUD & Use Gouraud shading of triangles \\ 
             & FLAT    & Use flat shading of triangles \\ \hline
scale\_scope  & GLOBAL\_SURFACE & Scaling global over each surface \\ 
             & GLOBAL\_FRAME   & Scaling global over each frame of data \\
             & GLOBAL\_GLOBAL  & Scaling global over all data \\
             & LOCAL          & Scaling local to each frame and surface\\
             & USER           & Scaling by user-supplied values (-pl -ph)
             \\
\hline
scale\_model  & LINEAR & Use linear scaling of contours \\
             & LOG    & Use logarithmic scaling \\
             & EXP    & Use exponential scaling \\
             & LAB7   & Use logarithmic in 7 levels scaling \\
             & LAB13  & Use logarithmic in 13 levels scaling \\ 
\hline
scale\_mapping & SYMMETRIC & scale symmetrically around both side of zero\\
              & SEPARATE  & scale separately for + and - data values\\
              & TRUE\_MAP  & scale from most - to most + values\\
\hline
color\_map     & RG       & Use full red-to-green colour map \\
              & RG2      & Use red-and-green (2-colour) colour map \\
              & FULL     & Use blue-to-red (full) colour map \\
              & BTW      & Use black-to-white colour map \\
              & WTB      & Use white-to-black colour map \\
\hline
lead\_marking\footnotemark 
              & LEADS          & mark nodes with lead (channel)
                                                             numbers\\ 
              & NODES          & mark nodes with node numbers\\
              & VALUES         & mark leads with potential values\\
              & CUBES          & mark leads with spheres/cubes \\
              & MINMAX\_LABELS & mark extrema with lead numbers \\
              & MINMAX\_CUBES  & mark extrema with cubes \\
              & SCALAR         & mark the selected scalar node \\
\hline
num\_cols      & value    & number of colours to use in shade display\\ \hline
num\_conts     & value    & number of contour levels in display\\ \hline
draw\_bbox     & TRUE/FALSE & set bounding box on or off \\ \hline
datafile\_path & pathame  & alternate path to the .pak/.raw files 
in .tsdf file. \\ \hline
geomfile\_path & pathame  & alternate path to the .geom files 
in .tsdf file. \\ \hline
report\_level  & value    & level of error reporting ( 0--3 ) \\ \hline
\end{tabular}
\end{center}

\footnotetext{options are cumulative} 

Note that these parameters and values are not case sensitive and that they
can all be overridden during execution of the program, typically via the
mouse menu (right mouse button).  See section~\ref{sec:scaling} for details
on the different scaling options.  The list of parameters possible will
also certainly grow with the program.

To save the current settings in a file, there is an option in the main menu
of \map{} which dumps all the settings to the file {\tt ./.map3drc}, that
is, the file {\tt .map3drc} in the current directory.  That way, when you
start the program again from that directory, the same settings will be loaded.
The {\tt .map3drc} file is just a normal text file, but like all
``rc''-files, it is hidden from the {\tt ls} command unless you add the
{\tt -a} option (the alias {\tt la} is set up by default to do a long
listing of all files, including hidden files).  Dumping a copy of the
settings is also the easiest way to see what settings are currently being
maintained by \map{} and also forms the best starting point to setting up
your own customized default settings files.

