% -*-latex-*-
% Document name: output.tex
% Creator: Rob MacLeod [macleod@vissgi.cvrti.utah.edu]
% Last update: September 2, 2000 by Rob MacLeod
%    - created
%%%%%%%%%%%%%%%%%%%%%%%%%%%%%%%%%%%%%%%%%%%%%%%%%%%%%%%%%%%%%%%%%%%%%%
\section{Output from \map{}}

\subsection{Capturing images for animation, printing, or photos/slides}
\label{sec:output} 

While screen images are lovely to look at, we need to be able to get the
output from the screen to some transportable medium like paper,
video tape, or film.  This section describes some of the methods available
for this process.

\subsubsection{Postscript dump}
\label{sec:psgl}

With the addition of the psgl software (after significant modification), we
are able to generate postscript files from map3d.  There are still some
restrictions and bugs in the system, but the essentials work and I
encourage its use.  To create a postscript file, run \map{} and set up the
view you want.  Then push Control-Shift-PrntScreen, all at the same time.
You will see some activity in the window you started \map{} from, and the
name of the output file will be listed somewhere in the output.  

You can view the resulting postscript file with {\em xpsview\/} or {\em
gv\/} and print it directly using the {\em printit\/} command.  The psgl
software produces proper postscript files, not images dumped to postscript
so the output quality is excellent.

At the moment, the system defaults to producing true colour postscript but
the option exists to convert the image to black lines (or gray shades) on a
white page.  To select this, go to the Shading/Contours menu or the Colour
Maps menu and select Toggle Postscript background.  Dashed lines are also
printed by psgl, if you have previously selected the ``Dashed lines for
negative values'' option in the Colour Map menu.

Problems and known bugs include the fact that numbers that a clipped out of
the view, appear to the right had side of the image, overwriting themselves
and some potentially useful stuff on the page.  This is a bug on the
underlying SGI software and I am still not clear what the best fix will
be.  At present, the only option is to edit the postscript directly.
Another weakness is that postscript does not perform Gouraud shading and so
coloured surface facets come out in a constant colour.  This too, we are
working on. 

\subsubsection{Image capture}
\label{sec:capture}
 
There are no standard provisions in GL for generating output from the
images generated by \map{}.  However, using an SGI utility (the same one
used in the program {\em snapshot\/}) it is, in fact, possible to capture the
contents of any area of the screen and save it as a file.  Once preserved,
this file can be viewed later, either by itself or as part of a sequence of
images, with an animation program called {\em movie\/}. The contents of the
file can also be converted either into other image formats for transport to
other computers or conversion to Postscript for generation of hard
copy.

To capture an image using \map{} simply set the image you want to preserve
and hit the ``w''-key. There will be a pause of between 5--30~s and then
the workstation should beep and a line will appear in the control window
telling you where the image has been stored.  Filenames for image storage
are generated automatically, using either the name of the geometry files
(if no data is present), the name of the current potential (or gradient)
file if data has been read into the program, or the filename specified with
the {\tt -if} option. Appended to this base filename
are sets of four digits, denoting the frame number currently in
the display, starting with ``0001''.  Thus, for
example, if the geometry files `{\tt daltorso.pts} and {\tt daltorso.fac}
are being displayed, with data from the file {\tt daltorso-pot004.pot}
(which was read as the fourth frame of potential data), the first file
produced would be {\tt daltorso004-0000.rgb}.  Note the {\tt .rgb} file
extension, standard for this sort of file, which is also referred to as an
``r-g-b'' file.  If, on the other hand, the user started \map{} with the
string {\tt -if movierun}, then the images files will be named {\tt
movierun-0001.rgb}, {\tt movierun-0002.rgb}, \etc{}.

The screen area captured in this mode is the smallest rectangle that
contains all the windows currently managed by the current invocation of
\map{}.  This often requires with careful placement of the windows or
setting the background window for the display to black or something that
matches the background of the \map{} display.

To view a file, or a set of files again on the screen, use the {\em movie}
program, or the {\em moviemaker}. Simply call the program and append to it
the name(s) of the files you want to view. To view a set of image files
with a common root, {\em e.g.,} {\tt daltorso000-001.rgb} to {\tt
daltorso000-010.rgb}, use the calling sequence {\tt
movie~daltorso000-*.rgb}.  The result is an animated sequence, which can be
stepped through, looped, or 'swung' back and forth. Once the images are
loaded, control of {\em movie\/} is with the right-hand mouse button,
pull-down menu.  (See the man page on {\em movie\/})

\subsubsection{Conversion of image files}
\label{sec:image-conversion} 

A frequent task when capturing images from the screen of a \map{} session
is the conversion of image formats from the native ``RGB'' mode of the SGI
to something more portable.  While the manual already describes the
necessary steps in performing such conversions (see
section~\ref{sec:output}), there is now a utility (a shell script, written
by \rob{}\footnote{For copies of the source code, contact \rob{} via email}
) available to perform a variety of conversions in a flexible manner.  The
utility is called ``rgbto'' and it has the following usage:

\begin{verbatim}
  Usage: rgbto [-i -c -b -a] infilename 
   -i to invert image colors (default is to leave alone) 
   -c to leave in colour (default is black/white) 
   -b to use black and white (default is gray shades) 
   -a to product (Apple) pict file output (default is postscript)
\end{verbatim}

The input is assumed to be one or more rgb files, and the output is either
in postscript or pict format.  The controls allow for the conversion from
colour to grey shades or black and white (by a simple thresholding), as
well as an inversion of the colours.  It will accept more than one filename
as input so that invocations such as, for example,\\
%
\begin{verbatim}
   rgbto -ca *.rgb
\end{verbatim}
%
are possible.

%%%%%%%%%%%%%%%%%%%%%%%%%%%%%%%%%%%%%%%%%%%%%%%%%%%%%%%%%%%%%%%%%%%%%%

\subsection{Video output from \map}
\label{sec:video} 

With the frame buffer card in the SGI, and the video
local area network (V-LAN) hardware, we can generate animation
sequences on video tape from \map.  Control of the sequence, which may be a
geometrical movement of objects in the display, or a series of sets of data
on a stationary geometry (or both concurrently), is by the
program {\em animator\/} described elsewhere.  In effect, {\em animator\/}
submits commands to the event loop of \map{} and then captures the images
from the GL window, moves them to the frame buffer, and then to video tape.

The most obvious feature of video recording in \map{} is that the user can
(actually must) select a window somewhere on the screen that defines what
portion of the display is actually captured for video.  If the -v option is
used to launch the program, the user will see a red, rectangular box,
together with a smaller, white text window with instructions.  When the
left mouse button is held down, moving the mouse now moves the rectangular
video box; when the middle mouse button is pushed the location of the video
box is fixed.  The region within the innermost rectangle fairly accurately
represents what will make it to the video screen.  To relocate the video
outline, return to the white text window and click the left mount button.

\subsection{Additional video controls}

With the separation of animation from video output and the desire to have
thicker lines for photos without the video colour scheme, \map{} now offers
independent control of the various aspects of what was formerly video
control.  There is now a menu for video control, which includes options to
turn the video window on (similar to the H-key), to toggle video colours
(similar to the V-key) and to set the line width that is used.

%%%%%%%%%%%%%%%%%%%%%%%%%%%%%%%%%%%%%%%%%%%%%%%%%%%%%%%%%%%%%%%%%%%%%%

\subsection{Photographing from the Display}
\label{sec:photo} 

There are now too different ways to convert images from \map{} (or any
other output on the screen of the SGI terminal) into photographs or slides.
The first, and most direct, is to set the camera up in front of the screen
and blast away.  The second involves grabbing images from the screen,
converting them to a form understood by a Macintosh and then reading (and
editing) them with an appropriate Mac program.

\subsubsection{Direct photography from the screen}

This process is simple and actually works very well (well enough to make
the cover of journals) if you proceed as follows:

\begin{enumerate}
  \item Use a tripod and the motor driven Canon cameras (we have two
        identical cameras for this purpose).
  \item Square the display screen so that it is parallel to the wall, turn
        the intensity up to near maximum, and
        clean the paw prints from the glass with glass cleanser.
  \item Either use the remote plunger or the delayed self-exposure
        mechanism to work the shutter.  Otherwise vibrations from your
        touching the shutter will blur the images.  Select the 2-second
        delay for minimal waiting time between shots.
  \item Most pictures can be taken at the 2-second exposure time.  Only
        with very bright images is it necessary to reduce the time and I
        have never needed a longer time.
  \item Select manual operation of the camera --- the light meter will be
        correct only for the case of a relatively bright, evenly illuminated
        image. 
  \item Set the f-stop at least one full f-stop larger (smaller aperture)
        than what the light meter suggests for anything
        but bright images that are very consistently illuminated.  I
        usually take three exposures starting at this setting and moving in
        full-f-stop steps toward larger f-stop values.  In the case of
        contour line images, this usually means starting at 3.5 and
        shooting again at
        5.6 and 8.3.  For shaded images, the f-stop will be larger to
        start with and will depend on the colour scheme used.
\end{enumerate}

Some tips that I have found useful for making photos are as follows:
%
\begin{itemize}
  \item You can keep a low level of room background light during photography
        from the screen; complete darkness is {\bf not} required, just
        avoid glare off the surface of the display.  I
        usually keep the lights from the opposite wall of the graphics lab
        on at a low level, enough to see my way around during the shooting.
  \item If you have the choice, keep the image towards the center of the
        screen and do not let it fill the entire screen.  Also, avoid having
        straight lines near the edge of the image as they will appear
        curved in the resulting photo (turn the bounding box off).
  \item Select a black background for the screen (under the root menu of
        the windowing system, or the ToolChest/Windows/Set Background menu
        selection).  If this doesn't seem to work, enter the command:\\
        \begin{verbatim}
                    xsetroot -solid black
        \end{verbatim}
  \item Iconify all other windows or at least shrink them and move them out
        of sight.
  \item Select borderless windows for map3d (-b option) to get rid of ugly
        window borders.  Borderless windows can still be moved and resized by
        holding the Alt-key down and pressing the right mouse button while
        the cursor is in the window you want to alter.
  \item Be imaginative --- map3d permits lots of kludging and the results
        are worth the effort.
\end{itemize}

\subsubsection{Photography via the Macintosh}

The method here is very similar to the image capture and conversion
described in sections~\ref{sec:capture} and~\ref{sec:image-conversion},
except that instead of converting the image to Postscript, we convert it to
a form that the Macintosh can read.  The form we have used with best
success so far is the ``pict'' file, a native Apple format that can be read
and written by most Macintosh applications.  To convert the .rgb file {\tt
filename.rgb} produced by snapshot into the pict file {\tt filename.pict},
use the rgbto command with the -a option, \eg{}
%
\begin{verbatim}
         rgbto -a filename.rgb 
\end{verbatim}

Now the file is in a form which is directly readable by the Macintosh.  The
file still resides on the SGI but can either be copied to the Mac using ftp
(accessible from either {\em NCSA Telnet\/} or {\em Versaterm\/}), or
the Macintosh program fetch.

For more information on the hard- and software in the Graphics Lab, talk to
Phil or see the documentation he is preparing on these facilities.

