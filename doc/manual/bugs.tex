% -*-latex-*-
% Document name: bugs.tex
% Creator: Rob MacLeod [macleod@vissgi.cvrti.utah.edu]
% Last update: September 4, 2000 by Rob MacLeod
%    - created from old version of the manual
% Last update: Mon Jul 23 13:28:30 2001 by Rob MacLeod
%    - release 5.2
% Last update: Tue Mar 30 13:28:30 2004 by Bryan Worthen
%    - release 6.0
% Last update: Wed Jun 30 13:28:30 2004 by Bryan Worthen
%    - release 6.1
% Last update: Sat Feb 17 23:26:58 2007 by Rob Macleod
%    - version 6.5
%%%%%%%%%%%%%%%%%%%%%%%%%%%%%%%%%%%%%%%%%%%%%%%%%%%%%%%%%%%%%%%%%%%%%%

\section{BUGS}
\label{sec:bugs}

Despite all our efforts, they creep in when we are not looking.  If
there are any you would like exterminated, please send email to
\htmladdnormallink{map3d@sci.utah.edu} {mailto:map3d@sci.utah.edu} (we
accept all foreign currency in large denominations, bicycle parts, assorted
outdoor gear, but no credit cards).  You can also go to the 
\htmladdnormallink{bugzilla}{software.sci.utah.edu/bugzilla} at the SCI
Institute and log the bug there.  The program will keep sending us annoying
emails until the problem is resolved.

Here is a short list of those we know about and are currently addressing:

\begin{itemize}
  \item When adding surfaces or updating time series data interactively,
    there can be surprises with scaling.  The easy solution is to adjust
    the scaling in some small way and the resulting update usually takes
    care of the problem.
\end{itemize}

%%% Local Variables: 
%%% mode: latex
%%% TeX-master: "manual"
%%% End: 
