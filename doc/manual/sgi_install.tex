% -*-latex-*-
% Document name: sgi_install.tex
% Creator: Bryan Worthen [worthen@sci.utah.edu]
% Last update: Fri Mar 26 20:00:00 2004 by Bryan Worthen
%    - release 6.0
%%%%%%%%%%%%%%%%%%%%%%%%%%%%%%%%%%%%%%%%%%%%%%%%%%%%%%%%%%%%%%%%%%%%%%
\subsection{SGI Installation Instructions}
\label{sec:sgi-install}

The SGI installation is relatively straightforward.  You'll need to 
download \map{}'s dependencies, then download \map{} itself.

\subsubsection{SGI Dependencies}
There are two phases to this part.  First we need to get GTK+ and its
dependencies.  The easiest way to do this is for SGI is to run a browser
with root access and go to SGI's freeware site at \htmladdnormallink
{\texttt{http://freeware.sgi.com}} {http://freeware.sgi.com}.  On that page
near the bottom there should be a link to a prereq calculator.  Click on
that link, and in the Freeware Product box, select \verb|fw_gtk2+| (make
sure you don't select \verb|fw_gtk+|).  Submit the query.  You should see a
list similar to this:
\begin{verbatim}
  fw_atk [relnotes] [prereqs] [install] 
  fw_expat [relnotes] [prereqs] [install] 
  fw_freetype2 [relnotes] [prereqs] [install] 
  fw_gettext [relnotes] [prereqs] [install] 
  fw_glib2 [relnotes] [prereqs] [install] 
  fw_libjpeg [relnotes] [prereqs] [install]
  fw_libpng [relnotes] [prereqs] [install]
  fw_libz [relnotes] [prereqs] [install]
  fw_pango [relnotes] [prereqs] [install]
  fw_tiff [relnotes] [prereqs] [install] 
\end{verbatim}
To install them, click on the install link one by one (and follow the 
instructions in the dialog boxes). 

IMPORTANT - when you install libz - it will mention something about a security
library being removed.  When you install libz, allow it to do this.  On the 
subsequent libraries, it will mention that the security package conflicts 
with libz, on these packages, have it continue without installing the
security package.

Install them in the order:

\begin{verbatim}
    gettext
    expat
    freetype2
    atk
    glib2
    pango
    libjpeg
    libtiff
    libz
    libpng
\end{verbatim}

After you've done all of these, click on the alphabetical link, and click on
the install buton that corresponds to \verb|libgtk2+-2.0.6|.

If for some reason, the prereq calculator isn't there or isn't working, go
to the alphabetical index and install the above in the order specified.


The next part is to download gtkglext, the library that supports OpenGL for
GTK widgets.  If your GTK version is 2.0.6 (you can find out by looking at
\texttt{gtkversion.h} which will be where you installed gtk (normally
\texttt{/usr/freeware/include/gtk-2.0/gtk/gtkversion.h}), and look for
\verb|GTK_MAJOR_VERSION|, \verb|GTK_MINOR_VERSION|, and
\verb|GTK_MICRO_VERSION|.  There will be numbers on the same lines as each
of these, and if you put them together it will be something like 2.0.6).
If you are using this version download \texttt{gtkglext-sgi.tar.gz} from
the \map{} download page
\htmladdnormallink{\texttt{http://www.sci.utah.edu/software/map3d.html}}
{http://www.sci.utah.edu/software/map3d.html} and follow these
instructions:

\begin{verbatim}
    cd <download directory>
    gunzip gtkglext-sgi.tar.gz
    tar xf gtkglext-sgi.tar
    cp libgtkglext-x11-1.0.so.0 /usr/local/lib
    cp libgdkglext-x11-1.0.so.0 /usr/local/lib
\end{verbatim}

You can copy them to some directory other than \texttt{/usr/local/lib} if
you wish.

If this doesn't work, you will need to download the gtkglext source and
compile it yourself (don't worry---if your gtk is properly set up, this
will be very easy).  Download the sources from Source Forge
\htmladdnormallink{\texttt{http://sourceforge.net/projects/gtkglext}}
  {http://sourceforge.net/projects/gtkglext} and follow these
  instructions:

\begin{verbatim}
    cd <download directory>
    gunzip gtkglext-1.0.6.tar.gz
    tar xf gtkglext-1.0.6.tar
    cd gtkglext-1.0.6
    configure
    make
    make install
\end{verbatim}


If you don't want these to end up in /usr/local/lib, you need to

\begin{verbatim}
configure --prefix=<dir>
\end{verbatim}

where \texttt{dir} is where you want to put the libraries (The libraries
will be in \texttt{dir/lib}).

\subsubsection{SGI Executable}

Download the \texttt{map3d-6.2-irix.tar.gz} file from the \map{} download
page and unzip it to a directory of your choice.  We will call that
\texttt{RUN-DIR}. This is the directory from which you will run \map{}.

To run \map{}, you will need to make sure that all the libraries are in
your \texttt{LD\_LIBRARY\_PATH} environment variable.  For this we will
assume that your gtk libraries are in \texttt{/usr/freeware/lib32} and your
gtkglext libraries are in \texttt{/usr/local/lib}.  Do the following:

\begin{verbatim}
    tcsh users:
    setenv LD_LIBRARY_PATH /usr/freeware/lib32:/usr/local/lib:$LD_LIBRARY_PATH
\end{verbatim}
%
or
%
bash users:
\begin{verbatim}
    export LD_LIBRARY_PATH=/usr/freeware/lib32:/usr/local/lib:$LD_LIBRARY_PATH
\end{verbatim}
%
you might want to put this line in your \texttt{.cshrc} or
\texttt{.profile} file to avoid having to run this multiple times.

%%% Local Variables: 
%%% mode: latex
%%% TeX-master: "manual"
%%% End: 
