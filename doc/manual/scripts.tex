% -*-latex-*-
% Document name: scripts.tex
% Creator: Bryan Worthen 
% Last update: June, 2017 
%    - created
%%%%%%%%%%%%%%%%%%%%%%%%%%%%%%%%%%%%%%%%%%%%%%%%%%%%%%%%%%%%%%%%%%%%%%

\section{Script Files}
\label{sec:scripts} 

Using script files to control map3d has numerous advantages, for example:

\begin{enumerate}
  \item complex layouts and specifications can be created and then held
        for later reuse,
  \item execution of the program reduces to a single word that starts the
        script,
  \item scripts are shell programs and can include logic and computation
        steps that automate the execution of the program; the user can even
        interact with the script and control one script to execute many
        different actions.
\end{enumerate}

\subsection{What are script files?}

A script files are simple programs written in the language of the Unix
shell.  There are actually several languages, one for each type of shell,
and the user is free to select.  At the CVRTI we have decided to use the
Bourne shell for script programming (and the Korn shell for interactive
use) and so all scripts will assume the associate language conventions.
The shell script language is much simpler to use and learn than a complete,
general purpose language such as C or Fortran, but is very well suited to
executing Unix commands; in fact, the script files consist mostly of lists
of commands as you might enter them at the Unix prompt.  Even more simply,
a script file can consist of nothing more than the list of commands you
would need to type to execute the same task from the system prompt.

\subsection{Automatically generating script files}
The easiest way to make a script file is to have \map{} do it for you.  Open a set
of surface, and arrange things how like them, and select ``Script file'' or ``Windows batch file'' 
Save menu.
In the script section of the dialog, select a filename and the type of script file, and 
push ``save''.  This will generate a script file that, when run, will open \map{} very closely
(if not identical) to how you had it before.

\subsection{How to make script files manually}

Script files are simple text files and so are usually created with an
editor such as emacs.  You can, however, also generate a script file from a
program, or even another script.  But all script files can be read and
edited by emacs and this is the way most are composed.

To learn about the full range of possibilities in script files requires some
study of a book such as ``Unix Shell Programming'' by Kochan and Wood but
the skills needed to make \map{} script files are much more modest; any
book on Unix will contain enough information for this.  The instructions
and examples below may be enough for many users.

Here are some rules and tips that apply to script files:

\paragraph{Use a new line for each command }  This is not a requirement but
makes for simpler files that are easier to read and edit.  If the command
is longer than one line, then use a continuation character ``\\'', \eg{backslash}
\begin{verbatim}
         map3d -f geomfilename.fac \
               -p potfilename.tsdf \
              -cl channelsfilename
\end{verbatim}

{\bf Make sure that there are no characters (even blank spaces) after the
continuation character!!!}  This has to be the most frequent error when the
script file fails to run or stops abruptly.

\paragraph{Make script files executable }  Script files can be executed by
typing {\tt . scriptfile} but the simplest thing is to make then executable
files so that they work simply by typing their names. To do this, use the
{\tt chmod} command as follows:
\begin{verbatim}
         chmod 755 script_filename
\end{verbatim}

\paragraph{Use the .sh extension for scripts }  This convention makes it
easy to recognize shell scripts as such, but also invokes some editor help
when you edit the file in emacs.  The mode will switch to shell (much like
Fortran or C mode when editing programs with .f and .c extensions) and has
some automatic tabbing and layout tools that can be helpful.

\paragraph{Variables in scripts }  The shell script is a language and like
any computer language there are variables.  To define a variable, simply
use it and equate it to a value, \eg{}
\begin{verbatim}
         varname=2
         varname="some text"
         varname=a_name
\end{verbatim}

Do not leave any spaces around the ``='' sign or the command will fail and
set the variable to an empty string.

Once defined, the variables can be used elsewhere in the script as follows:
\begin{verbatim}
         geomdir=/u/macleod/torso/geom
         geomfile=datorso.fac
         datafile=dipole.tsdf
         map3d -f ${geomdir}/${geomfile} -p $datafile
\end{verbatim}

The curly braces are required when the variable name is concatenated with
other text of variable names but is optional otherwise.  To concatenate
text and variables you simply write them together (\eg{} {\tt
${geomdir}/${geomfile}.pts} concatenates the two variables with a ``/'' and
the extension ``.pts''.

\paragraph{Environment variables in scripts}

All the environment variables are available and can be set in the script.
For example:
\begin{verbatim}
         mydir=${HOME}
\end{verbatim}
sets the variable {\tt \$mydir} equal to the user's home directory.
Likewise,
\begin{verbatim}
         MAP3D_DATAPATH=/scratch/bt2feb93/
         export MAP3D_DATAPATH
\end{verbatim}
defines and ``exports'' the environment variable used by map3d to find
.pak files. 

\paragraph{A note on scripting on Windows}
Windows scripting is a little different.
\begin{description}
  \item Windows environment variables can be set as follows:
  \begin{verbatim}
  set VAR=VALUE
  \end{verbatim}
  and referred to as \%VAR\% later.
  \item The \map{} command must all be on one line - it will not accept \textbackslash as a continuation character
\end{description}

% -*-latex-*-
% Document name: script-examples.tex
% Creator: Bryan Worthen 
% Last update: June, 2017 
%    - created
%%%%%%%%%%%%%%%%%%%%%%%%%%%%%%%%%%%%%%%%%%%%%%%%%%%%%%%%%%%%%%%%%%%%%%

\subsection{Examples}

Below are some sample scripts, from simple, to fairly complex:

\paragraph{Set the geometry, data, and window size and location}

\begin{verbatim}
         map3d -f ${HOME}/torso/geom/dal/daltorso.fac \
               -as 100 500 300 700 \
               -p ${HOME}/maprodxn/andy3/10feb95/data/cooling.tsdf \
               -s 1 1000
\end{verbatim}

\paragraph{\texttt{map3d-tank1.sh}, included with this distribution}

\begin{verbatim}
       MAP3D=../map3d
       GEOM=../geom/tank
       DATA=../data/tank

       $MAP3D -nw -f ${GEOM}/25feb97_sock.fac \
               -p ${DATA}/cool1-atdr_new.tsdf@1 -s 1 1000 \
               -ch ${GEOM}/sock128.channels \
               -f ${GEOM}/25feb97_sock_closed.geom \
               -p ${DATA}/cool1-atdr_new.tsdf@2 -s 1 1000\
               -ch ${GEOM}/sock128.channels
\end{verbatim}

\paragraph{\texttt{map3d-tank2.sh}, included with this distribution}

\begin{verbatim}
       MAP3D=../map3d
       GEOM=../geom/tank
       DATA=../data/tank

       $MAP3D -nw  -f ${GEOM}/25feb97_sock.fac \
               -as 200 600 400 800 \
               -p ${DATA}/cool1-atdr_new.tsdf@1 -s 1 476 \
               -at 200 600 200 420 -t 9\
               -ch ${GEOM}/sock128.channels \
               -f ${GEOM}/25feb97_sock_closed.geom \
               -as 590 990 400 800 \
               -p ${DATA}/cool1-atdr_new.tsdf@2 -s 1 476 \
               -at 590 990 200 420 -t 126 \
               -ch ${GEOM}/sock128.channels
\end{verbatim}

\paragraph{\texttt{map3d-torso1.sh}, included with this distribution}

\begin{verbatim}
       MAP3D=../map3d
       GEOM=../geom/torso
       DATA=../data/torso

       $MAP3D -f ${GEOM}/daltorso.geom -p ${DATA}/dipole2.tsdf -s 1 6
\end{verbatim}

\paragraph{\texttt{map3d-torso2.sh}, included with this distribution}

\begin{verbatim}
       MAP3D=../map3d
       GEOM=../geom/torso
       DATA=../data/torso

       $MAP3D -f ${GEOM}/daltorsoepi.geom@1 \
               -p ${DATA}/p2_3200_77_torso.tsdf -s 1 200 \
               -f ${GEOM}/daltorsoepi.geom@2 \
               -p ${DATA}/p2_3200_77_epi.tsdf -s 1 200

\end{verbatim}


\paragraph{Set some environment variables, then layout the whole display}

\begin{verbatim}
       #!/bin/sh
       # A script for the spmag 1996 article
       #
       ######################################################################
       map3d=/usr/local/bin/map3d
       map3d=${ROBHOME}/gl/map3d/map3d.sh
       MAP3D_DATAPATH=/scratch/bt26mar91pack/
       export MAP3D_DATAPATH
       echo "MAP3D_DATAPATH = $MAP3D_DATAPATH"
       basedir=/u/macleod/maprodxn/plaque/26mar91
       $map3d -b -nw \
               -f $basedir/geom/525sock.geom \
               -as 150 475 611 935 \
               -at 150 475 485 610 -t 237 \
               -p $basedir/data/pace-center.tsdf@1 \
               -s 65 380 \
               -f $basedir/geom/525sock.geom \
               -as 476 800 611 935 \
               -at 476 800 485 610 -t 237 \
               -p $basedir/data/pace-center.tsdf@1 \
               -s 65 380 \
               -f $basedir/geom/525sock.geom \
               -as 150 475 176 500 \
               -at 150 475 50 175 -t 237 \
               -p $basedir/data/pace-center.tsdf@1 \
               -s 65 380 \
               -f $basedir/geom/525sock.geom \
               -as  476 800 176 500 \
               -at  476 800 50 175 -t 237 \
               -p $basedir/data/pace-center.tsdf@1 \
               -s 65 380 
\end{verbatim}

\paragraph{A note on scripting on Windows}
Windows scripting is a little different.
\begin{description}
  \item Windows environment variables can be set as follows:
  \begin{verbatim}
  set VAR=VALUE
  \end{verbatim}
  and referred to as \%VAR\% later.
  \item The \map{} command must all be on one line - it will not accept \textbackslash as a continuation character
\end{description}



%%%%%%%%%%%%%%%%%%%%%%%%%%%%%%%%%%%%%%%%%%%%%%%%%%%%%%%%%%%%%%%%%%%%


%%% Local Variables: 
%%% mode: latex
%%% TeX-master: "manual"
%%% End: 


%%%%%%%%%%%%%%%%%%%%%%%%%%%%%%%%%%%%%%%%%%%%%%%%%%%%%%%%%%%%%%%%%%%%


%%% Local Variables: 
%%% mode: latex
%%% TeX-master: "manual"
%%% End: 
